\documentclass{amsart}
\usepackage{hyperref}
\usepackage{mathtools}
\usepackage{amsmath,amscd,amssymb}
\usepackage{tikz-cd}
\usepackage{ulem}
\newtheorem{thm}{Theorem}[section]
\newtheorem{prop}[thm]{Proposition}
\newtheorem{lem}[thm]{Lemma}
\newtheorem{cor}[thm]{Corollary}
\theoremstyle{definition}
\newtheorem{definition}[thm]{Definition}
\newtheorem{example}[thm]{Example}
\newtheorem{axiom}[thm]{Axiom}
\newtheorem{note}[thm]{Note}
\newcommand{\R}{\mathbb R}
\newcommand{\Q}{\mathbb Q}
\newcommand{\N}{\mathbb N}
\newcommand{\Z}{\mathbb Z}
\newcommand{\C}{\mathbb C}
\newcommand{\F}{\mathbb F}
\newcommand{\B}{\mathcal B}
\newcommand{\T}{\mathcal T}
\newcommand{\A}{\mathcal A}
\newcommand{\D}{\mathcal D}
\newcommand{\Power}{\mathcal P}
\newcommand{\st}{\text{ s.t. }}
\DeclareMathOperator{\Perm}{Perm}
\DeclareMathOperator{\Sym}{Sym}
\DeclareMathOperator{\GL}{GL}
\DeclareMathOperator{\id}{id}
\DeclareMathOperator{\lcm}{lcm}
\DeclareMathOperator{\SL}{SL}
\newcommand\bigzero{\makebox(0,0){\text{\huge0}}}
\DeclareMathOperator{\Aut}{Aut}
\DeclareMathOperator{\Inn}{Inn}
\DeclareMathOperator{\Out}{Out}
\DeclareMathOperator{\Free}{Free}
\DeclareMathOperator{\St}{St}
\DeclareMathOperator{\Aff}{Aff}
\DeclareMathOperator{\Fix}{Fix}
\DeclareMathOperator{\Syl}{Syl}
\hypersetup{
	colorlinks=true,
	linktoc=all,
	citecolor=blue,
	filecolor=blue,
	filecolor=blue,
	linkcolor=blue,
	urlcolor=blue,
}
\begin{document}
\title{Notes for MATH 503 By Prof. M. Mazur}
\author{Pluto Wang}
\maketitle
\tableofcontents
This course is an introduction to group theory: the second course in the graduate algebra sequence.
\section{Jan. 26}
\begin{definition}
	Let $X$ be a set. A \emph{binary operation} on $X$ is a function $f:X\times X\to X$. We will denote $f(x,y)$ by $x\Box y$. A binary operation is said to be \emph{associative} if $(x\Box y)\Box z=x\Box(y\Box z)$.
\end{definition}
\begin{definition}
	A \emph{monoid} is a set $M$ with a binary operation $\cdot$ which is associative and such that $\exists e\in M\st e\cdot m=m\cdot e=m$ for all $m\in M$.
\end{definition}
\begin{prop}
	$e$ in the previous definition of monoid is unique.
\end{prop}
\begin{proof}
	Let $e_1$ be another element so that $e_1\cdot m=m\cdot e_1=m$ for all $m\in M$. Then $e=e_1\cdot e=e_1$.
\end{proof}
We can thus uniquely define such $e$ to be the \emph{identity} element or \emph{neutral} element of $M$.
\begin{example}
	The natural number $\N$ with addition is a monoid, and $e=0$.
\end{example}
\begin{definition}
	A \emph{group} is a monoid $G\st\forall a\in G\ \exists b\in G\st a\cdot b=e$.
\end{definition}
\begin{example}
	The natural number $\N$ with addition and $e=0$ is not a group. But the integers $\Z$ with addition and $e=0$ is a group.
\end{example}
\begin{prop}
	Let $G$ be a group. If $a\cdot b=0$, then $b\cdot a=e$.
\end{prop}
\begin{proof}
	We have $c\in G\st b\cdot c=e$. Then $a=a\cdot e=a\cdot(b\cdot c)=(a\cdot b)\cdot c=e\cdot c=c$.
	
	Hence, $b\cdot a=e$.
\end{proof}
This also shows that $b$ is unique of $a$. We call it the inverse of $a$ and denote it $a^{-1}$. 
\begin{definition}
	We say that $a,b$ \emph{commute} if $ab=ba$. In a group, this is the same as $aba^{-1}b^{-1}$.
\end{definition}
\begin{definition}
	The \emph{commutator} of $a\cdot b$ is $[a,b]=aba^{-1}b^{-1}$.
	
	Note that some books use $[a,b]=a^{-1}b^{-1}ab$ and, in general, they are different.
\end{definition}
\begin{definition}
	A group $G$ is \emph{commutative} or \emph{abelian} if any two elements commute; i.e., $ab=ba$ for all $a,b\in G$.
\end{definition}
In abelian group, we often use additive notation; i.e., denote the operation $+$, $e=0$, and $a^{-1}=-a$.
\begin{example}
	These are some examples of groups.
	\begin{enumerate}
		\item The trivial group: $\{e\}$ where $e\cdot e=e$.
		\item The integers $\Z$ with addition $+$.
		\item The real $\R$ with addition $+$.
		\item If $R$ is a ring, then $(\R,+)$ is an abelian group. Called the additive group of the ring $R$.
		\item If $R$ is a ring, the units of $\R$ is $\R^\times=\{a\in R: ab=1=ba$ for some $b\in R\}$. This is a ring with multiplication and is called the multiplicative group of $R$.
		\item If $K$ is a field, then the $n\times n$ matrices over $K$, $M_n(K)$, is a ring. Note that $M_n(K)^\times=\GL_n(K)$, the general linear group of degree $n$ over $K$.
		\item We have $\Z^\times=\{1,-1\}$. So, $\GL_2(\Z)=\{\begin{bmatrix}
			a&b\\c&d
		\end{bmatrix}:a,b,c,d\in\Z$ and $ad-bc=\pm 1\}$ as $\begin{bmatrix}
			a&b\\c&d
		\end{bmatrix}^{-1}=\frac{1}{ad-bc}\begin{bmatrix}
			d&-b\\-c&a
		\end{bmatrix}$.
	\end{enumerate}
\end{example}
\begin{definition}
	Let $X$ be a set. Then the symmetry group of $S$, $S(X)=\Sym(X)$ is the set of all bijections $X\to X$ with composition of functions as the binary operation and $e=\id:X\to X$ by $\id(X)=X$. The inverse of $f$, $f^{-1}$ is just the inverse function of $f$ (whose existence is guaranteed by bijectivity).
\end{definition}
\begin{example}
	Let $X=V$ be a vector space. Then $\GL(V)$ is the set of all linear bijections of $V$.
\end{example}
\begin{definition}
	Let $X=\{1,2,...,n\}$. The symmetry group or permutation group on $n$ letter is just $S_n=S(X)$.
\end{definition}
Consider $X=\{a,b\}$, then $S(X)=S_2$ consists of two element, the identity map $id$, and $f:X\to X$ by $f(a)=b$ and $f(b)=a$.
\begin{example}
	Consider a square $ab-cd$. Let $r$ be the action of rotating $90^{\circ}$ clockwise and $s$ be the action of reflecting along the axis  across $ab$ and $cd$. Then $D_4=\{1,r,r^2,r^3,r^4,s,sr,sr^2,sr^3\}$.
	
	Multiplication of two actions gives a new rotation or reflecting, for example, $sr(a)=d$, $sr(b)=c$, $sr(c)=d$, and $sr(d)=a$.
	
	Note that we observe $rs=sr^3$, and can thus write the multiplication table as following.
	\begin{equation*}
		\begin{tabular}{c|cccccccc}
			$\cdot$ & $1$&$r$&$r^2$ & $r^3$ &$s$ & $sr$ & $sr^2$& $sr^3$\\
			\hline
			$1$ & $1$&$r$&$r^2$ & $r^3$ &$s$ & $sr$ & $sr^2$& $sr^3$\\
			$r$ & $r$&$r^2$&$r^3$ & $1$ &$sr^3$ & $s$ & $sr$& $sr^2$\\
			$r^2$ & $r^2$&$r^3$&$1$ & $r$ &$sr^2$ & $sr^3$ & $s$& $sr$\\
			$r^3$ & $r^3$&$1$&$r$ & $r^2$ &$sr$ & $sr^2$ & $sr^3$& $s$\\
			$s$ & $s$&$sr$&$sr^2$ & $sr^3$ &$1$ & $r$ & $r^2$& $r^3$\\
			$sr$ & $sr$&$sr^2$&$sr^3$ & $s$ &$r^3$ & $1$ & $r$& $r^2$\\
			$sr^2$ & $sr^2$&$sr^3$&$s$ & $sr$ &$r^2$ & $r^3$ & $1$& $r$\\
			$sr^3$ & $sr^3$&$s$&$sr$ & $sr^2$ &$r$ & $r^2$ & $r^3$& $1$
		\end{tabular}
	\end{equation*}
\end{example}
\begin{definition}
	Let $G$ be a group. Then a \emph{subgroup} of $G$ is a subset $H\subseteq G\st e\in H$ and if $a,b\in H$ then $ab\in H$ and $a^{-1}\in H$.
\end{definition}
\begin{prop}
	With the above definition, the subgroup $H$ is also a group under the restriction of the operation on $G$ to $H$.
\end{prop}
Proof of this is left as an exercise to the reader.
\section{Jan. 28}
\begin{example}
	The following are examples of groups:
	\begin{enumerate}
		\item Let $X$ be a set. Then $S(X)=\Sym(X)=\{f:X\to X: f $ is a bijection$\}$ with function composition is the symmetry group on $X$.
		\item Take $X=\{1,...,n\}$. Then $S_n=S(X)$ is the symmetry (permutation) group on $n$ letter.
		\item Let $S$ be a ring. Then $\GL_n(S)=M_n(S)^\times$ is all invertible $n\times n$ matrices with entries in $S$. Note that $\GL_1(S)=S^\times$.
	\end{enumerate}
\end{example}
\begin{definition}
	$S$ with two binary operations $+,\cdot$ is a (unitary) ring if\begin{enumerate}
		\item $(S,+)$ is an abelian group
		\item $(S,\cdot)$ is a monoid
		\item $(a+b)\cdot c=a\cdot c+b\cdot c$ and $c\cdot (a+b)=c\cdot a+c\cdot b$.
	\end{enumerate}
\end{definition}
\begin{definition}
	Let $G$ be a group. Then $H\subseteq G$ is a subgroup if $e\in H$ and $\forall a,b\in H$, $ab\in H$ and $a^{-1}\in H$.
	
	Note that $e\in H$ follows from the closure under multiplication and inverse, given $H$ is nonempty.	
\end{definition}
\begin{example}
	Let $G$ be a group. Then $Z(G)=\{a\in G\st \forall g\in G\  ag=ga\}$ is the center of the group. As an exercise, check it is a subgroup.
\end{example}
	It is easy to see that $G$ is abelian iff $G=Z(G)$.
\begin{note}
	One objective in group theory is to understand all subgroups of a given group $G$. Unfortunately, this is, usually, not easy.
\end{note}
\begin{thm}
	A subset $S$ of $(\Z,+)$ is a subgroup iff $S=d\Z$ for some $d\geq 0$.
\end{thm}
\begin{proof}
	The ``if'' direction is obvious: every $S=d\Z$ is a subgroup.
	
	Let $S$ be a subgroup of $\Z$. If $S=\{0\}$, then $d=0$ has $S=d\Z$. Otherwise, $S$ has positive elements.
	
	Take the smallest positive element $d\in S$. Take $a\in S$, then $a=nd+k$ where $0\leq k<d$. But $k=a-nd\in S$ which is necessarily $0$ as $d$ being the smallest positive element in $S$ and thus $a\in d\Z$;i.e., $S\subseteq d\Z$.
	
	Since $d\in S$, so $d\Z\subseteq S$. Thus, $S=d\Z$.
\end{proof}

As an exercise, proove that $k\Z\cap m\Z=\lcm(k,m)\Z$.
\begin{prop}
	The intersection of any collection of subgroups of a group $G$ is also a subgroup.
\end{prop}
\begin{proof}
	Take $\{H_i\}_{i\in I}$ be a collection of subgroups of $G$. Then $\forall i\in I$, we have $e\in H_i$; i.e., $e\in \cap H_i$.
	
	Take $a,b\in\cap H_i$, then $\forall i\in I$, $a,b\in H_i$. Thus, $ab\in H_i$ and $a^{-1}\in H_i$. Therefore, $ab\in\cap H_i$ and $a^{-1}\in \cap H_i$.
\end{proof}
\begin{definition}
	Let $X$ be a subset of $G$. Then $\langle X\rangle$ is the intersection of all subgroups containing $X$, called the subgroup \emph{generated} by $X$.
\end{definition}
Informally, $\langle X\rangle$ is the smallest subgroup that contains $X$, but subsets might not be comparable under the partial order relation.
\begin{prop}
	Let $X$ be a subset of group $G$. Then $g\in\langle X\rangle$ iff $g=e$ or $g=x_1^{\epsilon_1}\cdot ...\cdot x_s^{\epsilon_s}$ for $x_1,...,x_s\in X$ and $\epsilon_i=\pm 1$ for all $i$. Note that it is necessary to list the disjunct $g=e$ as $X$ could be $\emptyset$, in which case, $\langle \emptyset\rangle=\{e\}$.
\end{prop}
\begin{proof}
	Let $T=\{x_1^{\epsilon_1}\cdot ...\cdot x_s^{\epsilon_s}: x_1,...,x_s\in X, \epsilon_i=\pm 1\}$ for $X\not=\emptyset$. Then, we have
	\begin{enumerate}
		\item $e=x^1x^{-1}\in T$.
		\item If $a,b\in T$, then $ab\in T$.
		\item If $a=x_1^{\epsilon_1}\cdot ...\cdot x_s^{\epsilon_s}\in T$, then $a^{-1}=x_s^{-\epsilon_s}\cdot ...\cdot x_1^{-\epsilon_1}\in T$.
	\end{enumerate}
	
	Therefore, $T$ is a subgroup. Now, if $H$ is a subgroup of $G$, then $X\subseteq H$ implies $T\subseteq H$. Therefore, $T=\langle X\rangle$.
\end{proof}

When $X=\{g\}$, then we often denote $\langle X\rangle =\langle g\rangle$, and it is equal to $\{g^i:i\in\Z\}$.
\begin{definition}
	Let $g\in G$. Then $g^n=\begin{cases}
		\overbrace{g\cdot ...\cdot g}^n&n>0\\
		e&n=0\\
		\underbrace{g^{-1}\cdot ...\cdot g^{-1}}_{-n}&n<0
	\end{cases}$
\end{definition}
	As an exercise, shoe that $g^m\cdot g^n=g^{m+n}$ and $(g^{m})^n=g^{mn}$ for all $m,n\in\Z$.
\begin{definition}
	Groups generated by one element are called \emph{cyclic groups}; i.e., $G=\langle g\rangle$ is cyclic.
\end{definition}
For example, $\Z=\langle 1\rangle$ and in $D_4$, $\langle r\rangle =\{1,r,r^2,r^3\}$.
\begin{note}
	\begin{enumerate}
	\item If $g^n\not=g^m$ for all $n\not=m$, then $\langle g\rangle$ is infinite.
	\item If $g^n=g^m$ for some $n>m$, then $g^{n-m}=e$.
	\item Let $k>0$ be the smallest $\st g^k=e$, then $e, g, g^2, ..., g^{k-1}$ are all different.
	\end{enumerate}
If $l\in\Z$, $l=ak+r$ where $0\leq r<k$, then $g^l=g^{ak+r}=e\cdot g^r=g^r$. So, $\langle g\rangle =\{e,g,...,g^{k-1}\}$.
\end{note}
\begin{definition}
	$G$ is \emph{finite} if $G$ has finitely many element; i.e., $|G|<\infty$. Otherwise, it is \emph{infinite}.
	
	$g\in G$ is of \emph{finite order} if $|\langle g\rangle|<\infty$.
	
	The order of $g\in G$ is the smallest $k\in\N\st g^k=e$.
\end{definition}
\begin{example}
	In $S_n$, take $f$ by $f(1)=2$, $f(2)=3$,...,$f(n-1)=n$,$f(n)=1$. Then, $f$ is of order $n$. We thus have $\langle f\rangle$ is a cyclic group of order $n$.
\end{example}
\begin{definition}
	A group $G_1$ is \emph{isomorphic} to group $G_2$ if there is a bijection $f:G_1\to G_2\st f(ab)=f(a)f(b)$.	
\end{definition}
\begin{note}
If $e_1\in G_1$ and $e_2\in G_2$ are identities. Then $e_2f(e_1)=f(e_1)=f(e_1e_1)=f(e_1)f(e_1)$, and so, $f(e_1)=e_2$.

Also, $e_2=f(aa^{-1})=f(a)f(a^{-1})$, and so, $f(a^{-1})=(f(a))^{-1}$.	
\end{note}
\begin{example}
Suppose that $\langle g\rangle$ is infinite. Then $f:\Z\to \langle g\rangle$ by $m\mapsto g^m$ is a bijection. Also, $f(a+b)=g^{a+b}=g^ag^b=f(a)f(b)$. So, $f$ is an isomorphism.	
\end{example}
Another example is given by $\{1,-1\}$ with multiplication and $\{0,1\}$ with addition. These are isomorphic and can be shown by their multiplication table.
\begin{equation*}
\begin{tabular}{c|cc}
$\cdot$&$1$&$-1$\\
	\hline
	$1$&$1$&$-1$\\
	$-1$&$-1$&$1$
\end{tabular} \text{ and }
\begin{tabular}{c|cc}
	$+$&$0$&$1$\\
	\hline
	$0$&$0$&$1$\\
	$1$&$1$&$0$
\end{tabular} 
\end{equation*}
\begin{example}
Consider $\R_{>0}$ with multiplication and $\R$ with addition. These are groups. Also, $\R_{>0}\subseteq \R^\times =\langle\R_{>0}\cup \{-1\}\rangle$. 

$a\mapsto e^a: (\R,+)\to (\R_{>0},\cdot)$ is an isomorphism.	
\end{example}
\begin{definition}
	Let $G$, $H$ be groups. A function $f:G\to H$ is a \emph{homomorphism} if $f(ab)=f(a)f(b)$.
\end{definition}
\section{Jan. 31}
\begin{definition}
	Let $G,H$ be groups. A function $f:G\to H$ is a homomorphism if $f(ab)=f(a)f(b)$ for all $a,b\in G$.
\end{definition}
\begin{note}
	\begin{enumerate}
		\item $f$ is a homomorphism $\implies$ $f(e_G)=e_h$ and $f(a^{-1})=f(a)^{-1}$ for all $a\in G$.
		\item $f$ is called a \emph{monomorphism} if $f$ is injective (1-to-1).
		\item $f$ is called an \emph{epimorphism} if $f$ is surjective (onto).
		\item $f$ is called an \emph{isomorphism} if $f$ is bijective; and $f^{-1}:H\to G$ is also an isomorphism.
	\end{enumerate}
\end{note}

If there is an isomorphism between $G$ and $G$, we write $G\cong H$ and consider $G,H$ ``the same.''

\begin{example}
	$G$ a group, $g\in G$. Then there is a homomorphism $f:\Z\to G\st f(n)=g^n$ for all $n$. $f$ is injective iff $g$ has finite order. 
\end{example}
\begin{example}
	If $X$ and $Y$ are sets and $|X|=|Y|$ then $S(X)\cong S(Y)$.
\end{example}
\begin{proof}
	Suppose $\phi:X\to Y$ is a bijection, then $S(X)\to S(Y)$ by $f\mapsto \phi f\phi^{-1}$ is an isomorphism.
\end{proof}
Note that if $|X|=n$, then $|S(X)|=n!$.
\begin{example}
	$R$ a commutative ring. Then $\det:\GL_n(R)\twoheadrightarrow R^\times$ is a homomorphism.
	$$\bigg\vert\begin{bmatrix}
a&&&\\
&1&&\bigzero&\\
&&\ddots&&\\
&\bigzero&&1&\\
&&&&1
\end{bmatrix}
\bigg|=a$$
\end{example}
\begin{example}
	For all $n$, for all $R$ a ring. Let $P:S_n\to\GL_N(R)$ be for $f\in S_n$, define $P_f=(a_{ij})$ where $a_{ij}=\begin{cases}
		1&\text{if }i=f(j)\\
		0&\text{if otherwise}
	\end{cases}$; i.e., $P_f$ has only one non-zero entry in every row and every column, and all non-zero entries are $1$. Such matrices are called permutation matrices.
	
	For example, let $f=\begin{pmatrix}
		1&2&3\\3&2&1
	\end{pmatrix}$, $g=\begin{pmatrix}
		1&2&3\\2&1&3
	\end{pmatrix}\in S_3$. Then $fg=\begin{pmatrix}
		1&2&3\\2&3&1
	\end{pmatrix}$.
	
	Note that $P_f=\begin{bmatrix}
		0&0&1\\0&1&0\\1&0&0
	\end{bmatrix}$, $P_g=\begin{bmatrix}
		0&1&0\\1&0&0\\0&0&1
	\end{bmatrix}$, $P_{fg}=\begin{bmatrix}
		0&0&1\\1&0&0\\0&1&0
	\end{bmatrix}$.
\end{example}
As an exercise, show that $P_{fg}=P_fP_g$.

In $S_n$, consider $r=\begin{pmatrix}
	1&2&\dots&n-1&n\\
	2&3&\dots&n&1
\end{pmatrix}$ and $s=\begin{pmatrix}
	1&2&\dots&n-1&n\\
	n&n-1&\dots&2&1
\end{pmatrix}$. Let $D_n=\langle r,s\rangle$. This is the dihedral group on regular $n$-gon.

$r$ is rotation by $\frac{2\pi}{n}$ clockwise, $s$ is reflection in perpendicular bisector of $\overline{1n}$, and $D_n$ is all rigid motions of regular $n$-gon.

As an exercise, show $rs=sr^{n-1}$, order of $r=n$, and order of $s=2$.

Note that $D_n=\{1,r,...,r^{n-1},s,rs,...,r^{n-1}s\}$.

When $n=5$, then $rs^3rs^4=rsr=sr^{n-1}r=s$. Note that $srs=r^{n-1}$. $D_n$ is called dihedral group of order $2n$.

\begin{example}
	Let $G$ be a group. For $g\in G$, define $L_g:G\to G$ by $a\mapsto g\cdot a$ (Left multiplication by $g$). Then $L_g$ is a bijection as $ga=gb\implies a=b$ and $g(g^{-1}a)=a$.
	
	We have that $L_g\in S(G)$, so we can define $\phi:G\to S(G)$ by $g\mapsto L_G$. Then $L_g\circ L_h(a)=gha=L_{gh}a$, so, this is an injective homomorphism. 
\end{example}
\begin{thm}[Caley]
	Every group is isomorphic to a subgroup of $S(X)$ for some set $X$.
\end{thm}
If $G$ is a group and $g\in G$. Define $C_g:G\to G$ by $C_g(a)=gag^{-1}$. Then, $C_g$ is a homomorphism as $C_g(ab)=gabg^{-1}=gag^{-1}gbg^{-1}=C_g(a)C_g(b)$. Also, $C_g$ is a bijection as $gag^{-1}=gbg^{-1}\implies a=b$ and $g(g^{-1}ag)g^{-1}=a$.

These forms a homomorphism $f:G\to \Aut(G)$ by $g\mapsto C_g$ where $Aut(G)$ is the group of all automorphisms of $G$ under compositions.

\begin{definition}
	Elements of the form $C_g$ are called \emph{inner automorphisms} and $C_g$ is called ``conjugation by $g$.''
\end{definition}
Note: $\Aut(\Z)=\{\id,x\mapsto-x\}$.

If $G=\langle X\rangle$ and $f,h:G\to H$ are two automorphisms. Show as an exercise that if $f(x)=h(x)$ for all $x\in X$, then $f=h$.

$\GL_2(\Z)$ is finitely generated, $(\Q,+)$ and $(\Q^\times,\cdot)$ are not.

Show as an exercise that if $f:G\to H$ is a homomorphism, then $f(G)$ is a subgroup of $H$.
\begin{definition}
	Let $A,B$ be subsets of $G$. Then $AB=\{ab:a\in A,b\in B\}$.
	
\end{definition}
\begin{definition}
	Let $G$ be a group. $A,B$ are subsets of $G$. Then
	\begin{enumerate}
		\item $AB=\{ab\ |\ a\in A,b\in B\}$.
		\item $A^{-1}=\{a^{-1}\ |\ a\in A\}$.
		\item $aB=\{a\}B=L_a(B)$
	\end{enumerate}
\end{definition}

Let $f:G\to G$ be a homomorphism. Then $H=f(G)\leq G$ and we have $f:G\twoheadrightarrow H\hookrightarrow G$.

\begin{definition}
	$f^{-1}(e)=\{a\in G:f(a)=e\}=\ker(f)$ is the \emph{kernel} of $f$.
\end{definition}
\begin{prop}
	The kernel of $f$ is a subgroup of $G$.
\end{prop}
\begin{note}
	$f(a)=f(b)\iff f(ab^{-1})f(a)f(b)^{-1}=e\iff ab^{-1}\in\ker(f)$. so, $f^{-1}(f(a))=a\ker(f)=\ker (f)a$.
\end{note}
\begin{definition}
	A subgroup $N$ of $G$ is \emph{Normal} if $aN=Na$ for all $a\in G$; alternatively, $aNa^{-1}=N$ for all $a\in G$.
	
	($N$ is normal iff $N$ is preserved by all inner automorphism)
\end{definition}
As an exercise, show that If $N\leq G$ and $aNa^{-1}\subseteq N$ for all $a\in G$, then $aNa^{-1}=N$ for all $a\in G$.
\begin{note}
We denote $N$ is a subgroup of $G$ by $N\leq G$ and $N$ is a normal subgroup of $G$ by $N\unlhd G$. 
\end{note}
\begin{example}
	\begin{enumerate}
		\item Every subgroup of an abelian group is normal.
		\item $H=\{e,s\}\subseteq D_4$ has $rH=\{r,rs\}=\{r,sr^3\}$ and $Hr=\{r,sr\}\not=rH$, so not normal.
		\item $N=\{e,r^2\}$ is normal in $D_4$ as $r^kNr^k=N$ and $sNs^{-1}=N$
	\end{enumerate}
\end{example}
Show as an exercise that $Z(D_4)=\{e,r^2\}$.

\begin{prop}
	If $G=\langle X\rangle$, $X\subseteq G$, then $N$ is normal iff $\forall s\in X\ sNs^{-1}\subseteq N$ and $s^{-1}Ns\subseteq N$.
\end{prop}
Consider $f:G\twoheadrightarrow H\subseteq G$. We observe that elements of $H$ are in bijective correspondence with subsets of the form $a\ker f$ since if $h\in H$ then $f^{-1}(h)=a\ker f$ for some $a\in G$.
\begin{definition}
	Let $K\leq G$. A subset of $G$ of the form $aK$ ($Ka$) is called a \emph{left (right) coset} of $K$ in $G$ for $a\in G$.
\end{definition}
\begin{prop}
	$c\in aK$ iff $aK=cK$
\end{prop}
\begin{proof}
	If $cK=aK$, then $c=c\cdot e\in cK=aK$.
	
	If $c\in aK$, then $c=ak$ for some $k\in K$. so, $cK=akK=a(kK)\subseteq aK$. Also, $a=ck^{-1}\in cK$, so $aK\subseteq cK$. Hence,$cK=aK$.
\end{proof}
\begin{cor}
	Two left (right) cosets either coinside or are disjoints; i.e., the left (right) cosets partition the group.
\end{cor}
Show as an exercise that $(aK)^{-1}=Ka^{-1}$.
\begin{definition}
	$[G:K]$ is the index of $K$ in $G$ which is the number of left (right) cosets of $K$ in G.
\end{definition}
\begin{prop}
	Suppose $G$ is finite, so $K$ is finite. For $a\in G$, $|aK|=|K|$, so all cosets have the same number of elements.
	
	So, $|G|=[G:K]|K|$.
\end{prop}
\begin{cor}
	$|K|||G|$ if $K\leq G$.
\end{cor}
\begin{cor}
	If $g\in G$, then the order of $g$ divides $|G|$.
\end{cor}
\begin{cor}
	$g^{|G|}=e$.
\end{cor}
\begin{thm}[Fermat's Last Theorem]
	$p$ a prime, if $p\not| a$ then $p|a^{p-1}-a$.
\end{thm}
\begin{note}
	$\Z/p\Z$ is a field. $|(\Z/p\Z)^\times|=p-1$, and $a\in (\Z/p\Z)^\times\implies a^{p-1}=e$.
\end{note}
\begin{prop}
	$N\unlhd G$ iff every left coset of $N$ is also a right coset.
\end{prop}
The proof is left as an exercise.

Consider $f:G\twoheadrightarrow H\subseteq G$. $H$ is in a bijection w/ cosets of $\ker f$; i.e., $h\leftrightarrow f^{-1}(h)$.

\begin{definition}
	$G/N$ is the set of all cosets of a nornal group $N\unlhd G$.
\end{definition}
\begin{note}
	We can consider $f:G\twoheadrightarrow H$. Then $N=\ker f$, $aN=f(a)$, $bN=f(b)$, so, $abN=f(a)f(b)=f(ab)$. Then, $G/N$ is a group isomorphic to $H$. 
\end{note}
\begin{definition}
	Multiplication on $G/N$ by $(aN)(bN)=(ab)N$. Need to check that if $aN=a_1N$, $bN=b_1N$, then $abN=a_1b_1N$.
\end{definition}
\begin{proof}
	We have $a_1=an_1$, $b_1=bn_2$. Then $a_1b_1=an_1bn_2$. $Nb=bN\implies n_1b=bn_3 \implies a_1b=abn_3n_2=abn_4\in abN$.
\end{proof}
As an exercise, show that $(aN)(bN)=(ab)N$ as sets.
\begin{prop}
	$(G/N,\cdots)$ is a group.
\end{prop}
\begin{proof}
	We have $[(aN)(bN)](cN)=(ab)NcN=(ab)cN=a(bc)N=aN[bNcN]$. $e=N$. $aN\cdot N=aN$. $(aN)(a^{-1}N)=aa^{-1}N=N$.
\end{proof}
We have a canonical map called the quotient map. $\phi:G\to G/N$ by $g\mapsto gN$. It is surjective and is a homomorphism. $\ker \phi=N$.
\begin{example}
	Let $G=\Z$. Consider $n\Z$ where $n\geq 0$. Then $\Z/n\Z=\{0+n\Z,1+n\Z,...,(n-1)+n\Z\}$.
	
	$(a+n\Z)+(b+n\Z)=ab+n\Z=(a+b\mod n)+n\Z$ and $(a+n\Z)(b+n\Z)=ab+n\Z$. So, $\Z/n\Z$ is a ring.
\end{example}
\section{Feb. 7}
\begin{thm}
Let $G$ be a group. $H\leq G$, then the following are euivalent
\begin{enumerate}
	\item $aH=Ha$ for all $a\in G$
	\item $aHa^{-1}=H$ for all $a\in G$
	\item $aHa^{-1}\subseteq H$ for all $a\in G$
	\item Every left (right) coset of $H$ is also a right (left) coset.
\end{enumerate}
If $H$ has these properties, then we call $H$ to be normal, denoted $H\unlhd G$.
\end{thm}
\begin{prop}
Let $H\leq G$. Suppose for any $a,b\in H$, $(aH)(bH)$ is also a left coset. Then $H\unlhd G$ and $(aH)(bH)=(ab)H$.	
\end{prop}
The proof is left as an exercise.
\begin{prop}
If $f:G\to K$ is a homomorphism, then $\ker f\unlhd G$.	
\end{prop}
\begin{definition}
	Let $N\unlhd G$, then $G/N$ is the set of all coset of $N$ in $G$.
	
	With multiplication defined as $(aN)(bN)=(ab)N$, this is well-defined and $G/N$ is a group called the \emph{quotient group} of $G$ by $N$.
	
	The map $\phi:G\to G/N$ by $g\mapsto gN$ is a surjective group homomorphism, called the \emph{quotient map} and $\ker\phi=N$.
\end{definition}
\begin{prop}
	Suppose $f:G\to H$ is a surjective homomorphism and let $K=\ker f$, $\phi:G\to G/K$ the quotient map.
	
	Then there is a unique homomorphism $\bar f:G/K\to H\st \bar f\phi=f$ and $\bar f$ is an isomorphism. \begin{tikzcd}
	G\arrow[twoheadrightarrow, r,"f"]\arrow[two heads, d,"\phi"']&H\\
	G/H\arrow[dotted, ur, "\bar f"']	
	\end{tikzcd}
\end{prop}
\begin{proof}
	If $\bar f$ exists, then $\bar f(aK)=\bar f\phi (a)=f(a)$. so, it is unique if exists.
	
	Define $\bar f(aK)=f(a)$. If $aK=bK$, then $a=bk$ for $k\in K$, so $f(a)=f(bk)=f(b)f(k)=f(b)$. Therefore, it is well-defined.	
\end{proof}
\begin{prop}
	\begin{enumerate}
		\item Intersection of any collection of normal subgroups of $G$ is still normal.
		\item If $x\subseteq G$ and $aXa^{-1}\subseteq X$ for all $a\in G$, then $\langle X\rangle$ is normal.
		\item If $N\unlhd G$ and $H\leq G$, then $NH=HN$ is a subgroup of $G$.
		\item If $N\unlhd G$ and $H\unlhd G$ then $NH=HN\unlhd G$.
		\item If $N\unlhd G$ and $H\leq G$, then $H\cap N\unlhd H$.
	\end{enumerate}
\end{prop}
\begin{proof}
	(1) $N_i\unlhd G$, $i\in I$. Then $a\cap N_ia^{-1}=\cap aN_ia^{-1}=\cap N_i$.
	
	(2) Let $N=\langle X\rangle$. Then $aXa^{-1}\subseteq X\subseteq N$. So, $\langle aXa^{-1}\rangle=a\langle X\rangle a^{-1}=aNa^{-1}\subseteq N$ and $\langle X\rangle_n=\langle \bigcap\limits_{a\in G}aXa^{-1}\rangle$, where $\langle \cdot\rangle$ is the smallest normal subset containing $\cdot$.
	
	(3) Let $nh=h(h^{-1}nh)=hn'\in HN$ so $NH\subseteq HN$. Similarly $HN\subseteq NH$. So, $NH=HN$.
	
	Note that $NH=\langle N\cup H\rangle$. $nh(n_1h_1)=nhn_1h^{-1}hh_1\in NH$ and $nh=h^{-1}n^{-1}=h^{-1}nhh^{-1}\in NH$.
	
	(4) $a(HN)a^{-1}=(aHa^{-1})(aNa^{-1})=HN$.
	
	(5) $h(N\cap H)h^{-1}=(hNh^{-1})\cap (hHh^{-1})=N\cap H$.
\end{proof}
\begin{thm}[First homomorphism theorem]
Let $\phi:G\to K$ be a surjective homomorphism and $f:G\to H$ a homomorphism $\st\ker f\subseteq \ker \phi$. Then there is a unique homomorphism $\bar f:K\to H\st \bar f\phi=f$. Also, $f(G)=\bar f(K)$ and $\ker\bar f=\phi\ker f$. \begin{tikzcd}
	G\arrow[r,"f"]\arrow[two heads, d,"\phi"']&H\\
	K\arrow[dotted, ur, "\bar f"']	
	\end{tikzcd}
\end{thm}
\begin{proof}
if $\bar f$ exists then $\bar f(k)=\bar f(\phi g)=f(g)$ for $\phi g=k$. So, $\bar f$ is unique if exists.

If $\phi(g_1)=\phi(g_2)=k$, then $g_1g_2^{-1}\in\ker\phi$ and so $g_1g_2^{-1}\in\ker f$. Therefore, $f(g_1)=f(g_2)$ and thus $\bar f$ is well-defined. Define $\bar f(k)=f(g)$ for any $g\in G\st \phi(g)=k$.	
\end{proof}
\begin{cor}
If $\phi:G\to G/N$ is a quotient map. and $N\in\ker f$, then $\ker \bar f=\ker f/N$.	
\end{cor}
\begin{thm}[Correspondence Theorem]
	Let $f:G\twoheadrightarrow H$ be a surjective homomorphism. Then
	\begin{enumerate}
		\item $K\leq G\implies f(K)\leq H$ ($K\unlhd G\implies f(K)\unlhd H$).
		\item $T\leq H\implies f^{-1}T\leq G$ and $\ker f\subseteq f^{-1}(T)$.
		\item If $K\leq G$, then $f^{-1}(f(K))=K\ker f$.
		\item If $T\leq H$, then $f(f^{-1}(T))=T$.	
	\end{enumerate}
To summarize: $T\mapsto f^{-1}(T):\text{subgroups of }H\to \text{subgroups of }G \text{ containing} \ker f$ is a bijective correspondence that preserves inclusion and intersection with normal subgroups corresponding to normal subgroups. In particular, if $f:G\to G/N$ is the quotient map, then subgroups of $G/N\leftrightarrow$ subgroups of $G$ containing $N$; i.e.,
$$N\subseteq K\subseteq G\leftrightarrow K/N\subseteq G/N$$ 
\end{thm}
\begin{thm}[Second homomorphism theorem]
If $K\unlhd G$, $H\leq G$, $A\unlhd H$. Then $KH\lhd G$, $KA\unlhd KH$ and the quotient map $\phi:KH\to KH/KA$ takes $H$ onto $KH/KA$ and the kernel is $(H\cap K)A$.

	Furthermore, $H/(H\cap K)A\cong KH/KA$ in a canonical way by $h((H\cap K)A)\mapsto h(KA)$. If $A=\{e\}$, then we have $H/H\cap K\cong KH/K$.
\end{thm}
\begin{thm}[Modular Law]
$G$ a group. $H,K,L$ subgroups of $G\st K\subseteq L$. Then $(HK)\cap K=(H\cap L)K$.	
\end{thm}
\section{Feb. 9}
\begin{thm}[Homomorphism Theorems]
The following are the four homomorphism theorems.
\begin{enumerate}
	\item \begin{tikzcd}
	G\arrow[r,"f"]\arrow[two heads, d,"\phi"']&H\\
	K\arrow[dotted, ur, "\bar f"']	
	\end{tikzcd} If $\phi:G\twoheadrightarrow K$ is a surjective homomorphism and $f:G\to H$ is a homomorphism s.t. $\ker\phi\subseteq\ker f$. Then there is a unique homomorphism $\bar f:K\to H\st \bar f\phi=f$. Hence $\bar f(k)=f(a)$ and $\ker\bar f=\phi(\ker f)$.
	\item Let $f:G\twoheadrightarrow H$ a surjective homomorphism then the assignment $K\to f(K)$ is a bijective correspondence between subgroups of $G$ that contain $\ker f$ and subgroups of $H$ which preserves inclusion, intersection, and normality.
	\item Let $N\unlhd G, H\leq G, A\unlhd H$. Then $H/(H\cap N)A\to NH/NA$ by $h(H\cap N)A\mapsto hNA$ is a group isomorphism. In particular, if $A=\{e\}$, then $H/H\cap N\cong NH/N$.
\begin{center}
	\begin{tikzcd}
		NH\arrow[dash, d,"\subseteq"']\arrow[r,"\phi"]&NH/NA\\
		H\arrow[two heads, dr, "\text{quotient}"']\arrow[ur, "\widetilde{\phi}"']&\\
		&H/(H\cap N)A\arrow[uu]
		
	\end{tikzcd} $\ker\widetilde\phi=H\cap NA=(H\cap N)A$
\end{center}
\item Let $K\unlhd G, H\unlhd G, K\subseteq H$. Then $G/H\to (G/K)/(H/K)$ by $gH\mapsto (gK)H/K$ is an isomorphism.
\end{enumerate}
\end{thm}
\begin{example}
	$G$ a group, $g\in G$. Consider $\phi: \Z\to G$ by $n\mapsto g^n$. Then $\ker\phi=m\Z$ where $m$ is the order of $g$. So, $\Z/m\Z\cong \langle g\rangle$
	
	Note that in $\Z/m\Z$, take $a\in \Z$. $a+m\Z$ generates $\Z/m\Z$ iff $\gcd(a,m)=1$.
\end{example}
\begin{example}
	$\det:\GL_n(K)\to K^\times$ is a surjective group homomorphism (for any commutative ring). Note that $\ker(\det)=\SL_n(K)$.
	
	Scalar notation: $aI, a\in K^\times$ form a normal subgroup of $\GL_n(K)$. This is the center.
	
	The quotient $\GL_n(K)/\{aI\}=PGL_n(K)\supseteq PSL_n(K)$.
	
	$$SL_n(\Z/p\Z)\supseteq (\Z/pZ)^\times I\subseteq GL_{p-1}(\Z/p\Z)\overset{\det}\to(\Z/p\Z)^\times$$ 
\end{example}
\begin{example}
	Consider the permutation group on $n$ letters. 
	$$S_n\hookrightarrow \GL_n(\Z)\overset{\det}\to\{1,-1\}=\Z^\times$$
	
	This induces $\pi:S_n\to \Z^{1,-1}$. 
	
	Note that $\pi$ is surjective as $\det\begin{bmatrix}
0&1&&&\\
1&0&&\bigzero&\\
&&1&&\\
&\bigzero&&\ddots&\\
&&&&1	
\end{bmatrix}
=-1$.

Here $\ker\pi=A_n$ is the alternating group. $[S_n:A_n]=2$ and $S_n/A_n\cong\{1,-1\}=\Z/2\Z$.
\end{example}
\begin{example}
Let $\phi:G\to \Aut G$ by $g\mapsto C_g$ where $C_g:a\mapsto gag^{-1}$.

Then $\ker\phi=Z(G)$ which is the center of $G$. $\phi(G)=\Inn G$ which are the inner automorphism on $G$. 	
\end{example}
As an exercise, show that $\Inn G\unlhd \Aut G$. $\phi C_g\phi^{-1}=C_{\phi g}$.
\begin{definition}
The outer automorphisms $\Out G=\Aut G/Inn G$.

$G$ is \emph{complete} if $G\to \Aut G$ is an isomorphism.

$G$ is \emph{simple} if $\{e\}$ and $G$ are the only normal subgroups of $G$.	
\end{definition}
\begin{example}
	$p$ a prime. Then $\Z/p\Z$ are simple. These are the only simple abelian simple groups. 	
\end{example}
\begin{prop}
$G$ a group. $N\unlhd G, K\unlhd G$. If $N\cap K=\{e\}$ then $nk=kn$ for all $n\in N, k\in K$.	
\end{prop}
\begin{proof}
Consider $nkn^{-1}k^{-1}$. On one hand, $nkn^{-1}\in K$	 and $k^{-1}\in K$, so it is in $K$. On the other hand, $n\in N$ and $kn^{-1}k^{-1}\in N$, so it is in $N$. Therefore, $nkn^{-1}k^{-1}\in K\cap N=\{e\}$. Therefore, $nk=kn$.
\end{proof}
Therefore, $NK=N\times K$ if $N,K\unlhd G$ and $N\cap K=\{e\}$.
\begin{definition}
	Given a collection of groups $(G_i)_{i\in I}$, we define $\prod\limits_{i\in I}G_i$ to be the set of all functions $f:I\to\cup G_i\st \forall i\in I f(i)\in G_i$ where $(g\star f)(i)=f(i)g(i)$, this is the groups called the \emph{product} of $G_i$.
	
	Note that this definition corresponds to the strings of $g_i$ where $f\leftrightarrow (g_i)$ s.t., $f(i)=g_i$.
\end{definition}
\begin{definition}
For all $i\in I$, we have a homomorphism $\alpha_i:G_i\to \prod G_i$ by $g\mapsto f$ where $f(j)=\begin{cases}
	e&j\not=i\\g&i=j
\end{cases}$

Also, we have $\pi_i:\prod G_i\to G_i$ by $(g_i)\mapsto g_i$.

Given $\phi_i:H\to G_i$, there is a unique $\phi:H\to \prod G_i\st \phi_i=\pi_i\phi$. for all $i$	
\end{definition}
Inside of $\prod\limits_{i\in I}G_i$, we have subgroups $\bigoplus\limits_{i\in I}G_i$; the direct sums of $G_i$ which consists of all those $f\st f(i)\not=e$ for at most finitely many $i$.
\begin{prop}
Given any collection $\phi_i:G_i\to A_i$ where $A_i$ abelian groups. There is a unique $\phi:\bigoplus\limits_{i\in I} G_i\to A\st \phi\alpha_i=\phi_i$ by $\phi((g_i))=\sum\phi_i(g_i)$.	
\end{prop}
\begin{example}
Suppose $\gcd(m,n)=1$, then $\Z/mn\Z\cong\Z/m\Z\times\Z/n\Z$ as we have $\langle m+mn\Z\rangle\cap\langle n+mn\Z\rangle =\{e\}$ where $\langle m+mn\Z\rangle =\Z/n\Z$ and $\langle n+mn\Z\rangle =\Z/m\Z$.	
\end{example}
As an exercise, show that\begin{enumerate}
\item $n=p_1^{k_1}\cdot ...\cdot p_s^{k_s}$, $\Z/n\Z\cong \Z/p_1^{k_1}\Z\times ...\times \Z/p_s^{k_s}\Z$.
\item $\Z/n\Z\times \Z/m\Z\cong \Z/\gcd(m,n)\Z\times\Z/\lcm(m,n)\Z$.	
\end{enumerate}


Consider $A=\bigoplus\limits_{i\in I}\Z$, then every element of $A$ can be uniquely written as $\sum m_ie_i=(m_i)$ for $m_i\in\Z$ and finitely many of them are not zero.

Let $G$ be an abelian group (we use additive notation). Then the elements $(g_i)_{i\in I}$ have the property that $G\cong \bigoplus\limits_{i\in I}\rangle g_i\langle$ is an isomorphism iff $\oplus\langle g_i\rangle=G$ ($\{g_i:i\in I\}$ generates $G$).

If $m_1g_1+...+m_sg_s=0$ then $m_1g_1,...,m_sg_s=0$.
\section{Feb. 11}
\begin{definition}
Let $G_i$, $i\in I$ be groups. Then $\prod_{i\in I}G_i=\{f:I\to\bigcup\limits_{i\in I}G_i: \forall i\in I\ f(i)\in G_i\}$.

A function $f$ is often denoted $(f_i)_{i\in I}$ where $f_i=f(i)$. We have $(f\star g)(i)=f(i)g(i)$.

There are projections: $\pi_i:\prod G_i\to G_i$ by $\pi_i(f)=f(i)$.

There are also embbedings: $e_i:G_i\to \prod G_i$ by $e_i(g)(j)=\begin{cases}
	e&j\not=i\\g&j=i
\end{cases}$.	
\end{definition}
\begin{definition}
The direct sum $\bigoplus\limits_{i\in I}G_i\subseteq \prod G_i$ of the groups $G_i$ consists of $f\st f(i)=e$ except for finitely many $i$.	
\end{definition}
\begin{prop}[Universal Property]
	Given an abelian group $A$ and homomorphisms $\phi_i:G_i\to A$, there is a unique $\phi:\bigoplus\limits_{i\in I}G_i\to A\st \phi e_i=\phi_i$ by $\phi((g_i))=\sum_i\phi_i(g_i)$.
\end{prop}
\begin{example}
	\begin{enumerate}
		\item $V$ is a vector space over a field $K$ then $(V,+)\cong \bigoplus\limits_{i\in I}K$ for some $I$.
		\item $K$ a field. Then $K$ contains either $\Q$ or $\Z/p\Z$ where $p$ is a prime as a subfield (it is called the prime subfield of $K$). $(K,+)\cong\begin{cases}
			\bigoplus\limits_{i\in I}\Q&\Q\subseteq K\\
			\bigoplus\limits_{i\in I}\Z/p\Z&\Z/p\Z\subseteq K
		\end{cases}$	
	\end{enumerate}
\end{example}
\begin{definition}
	$A$ abelian group, $(a_i)$, $i\in I$ some elements in $A$. The natural homomorphism $\phi:\bigoplus\limits_{i\in I}\langle a_i\rangle \to A$ by $(m_ia_i)\mapsto \sum\limits_{i\in I}m_ia_i$.
	
	1. $\phi$ is onto iff $A$ is generated by $\{a_i\}_{i\in I}$.
	
	2. $\phi$ is injective iff whenever $\sum_{i\in I}m_ia_i=0$, we have $m_ia_i=o$ for all $i\in I$.
	
	If $(a_i)$ has property 2, we say that $a_i$ are independent in $A$. If in addition they have property 1, we say they form a basis of $A$.
\end{definition}
\begin{example}
$\Z/6\Z\cong \Z/2\Z\times\Z/3\Z$, we have $\Z/6\Z \cong\langle3+6\Z\rangle\oplus \langle 2+6\Z\rangle$. So, $\{1+6\Z\}$ is a basis of $\Z/6\Z$ and $\{3+6\Z,2+6\Z\}$ is also a basis of $\Z/6\Z$.	
\end{example}
\begin{definition}
An abelian group $F$ is called \emph{free abelian} if it has a basis consisting of elements of infinite orders (then every element $\not= e\in F$ has infinite orders).
$$F\text{ is free abelian }\iff F\cong \oplus_{i\in I}\Z$$	
\end{definition}
\begin{cor}
	Every abelian group is a quotient of a free abelian group. An abelian group can be generated by $n$ elements iff it is a quotient of $\Z^n$.
\end{cor}
\begin{proof}
If $a_i$, $i\in I$ generates $A$, then the maps $\phi_i:\Z\to A$ by $i\mapsto a_i$ gives surjective homomorphism $\bigoplus\limits{i\in I}Z\twoheadrightarrow A$.	

If $A$ is generated by $n$ elements then we get $\Z^n\twoheadrightarrow A$. Conversely, if $\Z^n\twoheadrightarrow A$, then since $\Z^n$ is generated by $n$ elements, we have $A$ is generated by their images. 
\end{proof}
Idea: in order to understand $n$-generated abelian groups, we need to understand subgroups of $\Z^n$.
\begin{example}
	$n=1$, subgroups of $\Z$ are $k\Z$ where $k\geq 0$, so they are all cyclic.
\end{example}
\begin{prop}
Let $N\unlhd G$, if $N$ cam be generated by $s$ elements and $G/N$ can be generated by $t$ elements, then $G$ can be generated by $s+t$ elements.	
\end{prop}
\begin{proof}
Let $a_1,...,a_s$ generates $N$ and $b_1N,...,b_tN$ generates $G/N$. Consider $H=\langle a_1,...,a_s,b_1,...,b_t\rangle$. Note that $N\subseteq H$

Also, let $\pi:G\to G/N$, then $\pi(H)$ contains $b_1N,...,b_tN$. So, $\langle g_1N,...,g_tN\rangle\subseteq \pi(H)$. So, $\pi(H)=G/N$. By correspondence, $H=G$. 	
\end{proof}
\begin{cor}
A subset of $\Z^n$ can be generated by $n$-elements.	
\end{cor}
\begin{proof}
Induction on $n$. If $n=1$, $d\Z$ can be generated by $d$.

Define $K\leq \Z^n$, let $e_1,...,e_n$ be the standard basis.

$\Z\cong \langle e_1\rangle\subseteq \Z^n\overset{\pi}\twoheadrightarrow\Z^{n-1}$. Also, $K\cap \langle e_1\rangle \subseteq K\twoheadrightarrow\pi(K)$. Note that $K\cap \langle e_1 $ is a subgroup of $\langle e_1\rangle$, so it is cyclic. 

By induction, $\pi(k)$ can be generated by $n-1$ elements, and $\pi(K)\cong K/(K\cap\langle e_1\rangle)$. 	
\end{proof}
\begin{note}
Let $F$ be a free abelian group with basis $e_1,...,e_n$ and $A$ be subgroups generated by $w_1,...,w_m$ (we don't necessarily have $m\leq n$).

Now, $w_i=\sum\limits_{j=1}^n m_{i,j}e_j$ where $m_{i,j}\in\Z$. Let $M=(m_{i,j})$ a $m\times n$ matrix.

Pick $i\not=j$, 1. if we replace $w_i$ by $w_i+kw_j$ and keep the rest unchanged, then we get another generating set and the new matrix $M$ which is obtained from $m$ by adding $k\cdot j$th row to the $i$th row.

2. if we replace $e_j$ by $e_j-k\cdot e_j$ and keep the rest unchanged, then we get a new basis of $F$ and the corresponding $M$ is obtained from $M$ by adding $k\cdot j$th column to $i$th column of $M$.

3. Permuting $e_i$'s permutes the column and permuting $w_i$'s permutes the rows.

We start with $M$. Find the non-zero entry of the smallest absolute value of $M$ and permute, so it is the $1-1$ entry. Replacing $e_i$ by $-e_i$ we may assume that $k_{1,1}>0$.

Suppose $k_{1,1}\not|k_{i,1}$ for some $i$. Then $k_{i,1}=pk_{1,1}+r$ for $0<r<k_{1,1}$. Subtracting $p\cdot$ 1st row from $i$th and have $k_{i,1}=r<k_{1,1}$.

Repeat the process, then we have the resulting $\bar e_1,...,\bar e_n$ is a basis, $\bar w_1=k_{1,1}\bar e_1$ and $\{\bar w_2,...,\bar w_m\}\subseteq \rangle \bar e_2,...,\bar e_n\langle$.	
\end{note}
\begin{thm}
There is a basis $\{\bar e_1,...,\bar e_n\}$ of $F$ and $k_1|k_2|k_3|...|k_r\st k_1\bar e_1, ...,k_r\bar e_4$ generate $A$.	
\end{thm}
\begin{cor}
$A$ is free with basis 	$k_1\bar e_1, ...,k_r\bar e_4$.
\end{cor}
\begin{cor}
$F/A\cong \Z/k_1\Z\oplus...\oplus \Z/k_r\Z\oplus \Z^{n-s}$.	
\end{cor}






\section{Feb. 14}
\begin{thm}
Let $F$ be a free abelian group with basis of size $n$, and let $\{0\}\not=A<F$. Then there is a basis $e_1,...,e_n$ of $F$ and positive integers $k_1|k_2|...|k_s$ for some $s\leq n\st k_1e_1, k_2e_2,...,k_3e_3$ generate $A$. 
\end{thm}
The idea of the proof is to start with a basis $b_1,...,b_n$ of $F$ and generating set $w_1,...,w_v$ of $A$. Write $w_i=\sum_{j}m_{ij}b_j$ and consider $M=(m_{ij})$. By a sequence of operations of the form
\begin{enumerate}
	\item For $i\not=j$, replace $w_i$ by $w_i+kw_j$ for some $k\in\Z$.
	\item For $i\not=j$, replace $e_i$ by $e_i+kw_j$ for some $k\in\Z$.
	\item Permute the basis basis elements or the generators of $A$.
	\item Replace a basis element or generator by its inverse.
\end{enumerate}
transform the bases and generating set, so that the corresponding $M$ is $\begin{bmatrix}
k_1	& & \bigzero& \vline\\
\bigzero& \ddots& &\vline&\bigzero&\\
&&k_s&\vline\\
\hline
\bigzero&&\bigzero& \vline& \bigzero\\
\\
\end{bmatrix}$.

We often call the bases in the theorem a compatible choice of bases of $F$ and $A$.
\begin{cor}
	$A$ is free abelian. In general, a subgroup of any free abelian group is free abelian.
\end{cor}
\begin{thm}
Let $G$ be a finitely generated abelian group. Then $G \cong \Z/k_1\Z \oplus...\oplus \Z/k_r\Z\oplus \Z^t$ for some $1<k_1|k_2|...|k_r$ and $t\geq 0$.	
\end{thm}
\begin{proof}
Since $G$ is $n$-generated, then we have a surjective map $\Z^n\overset{\pi}\twoheadrightarrow G$. If $\ker(\pi)=A$, choose compatible basis $\{e_1,...,e_n\}$ of $\Z^n$ and $l_1e_1,...,l_se_s$ of $A$ so that $l_1|l_2|...|l_s$. Then we have $\Z/A\cong\Z/l_1\Z\oplus ...\oplus \Z/l_s\Z\oplus \Z^{n-s}$ and if we remove all $l_i=1$, we have the result.
\end{proof}
\begin{prop}
$\Z^n$ can not be generated by fewer than $n$ elements.	
\end{prop}
\begin{proof}
$\Z^n\subseteq \Q^n$ and if $e_1,...,e_k$ generates $\Z^n$ as abelian group, then $e_1,...,e_k$ span $\Q^n$ as $\Q$-vector space.

If $v\in\Q^n$ then $N\cdot v=\Z^n$ and thus $N\cdot v=\sum m_ie_i$, $v=\sum \frac{m_i}{N}e_i$. Therefore, $k\geq n$.)
\end{proof}
\begin{cor}
If $k\not=n$	then $\Z^k\not\cong\Z^n$.
\end{cor}
\begin{proof}
If $k<n$, then $\Z^k$ is generated by $k$ elements, but $\Z^n$ cannot be generated by $n$ elements.
\end{proof}
\begin{definition}
	The number of basis elements of a finitely generated abelian group $F$ is unique, and is called the \emph{rank} of $F$.
\end{definition}
Let $G\cong \Z/k_1\Z\oplus ...\oplus \Z/k_r\Z\oplus \Z^t$, where $1<k_1|k_2|...|k_r$. Then
\begin{enumerate}
	\item $\Z/k_1\Z\oplus ...\oplus \Z/k_r\Z$ are the elements of finite order, we call it the \emph{torsion} of $G$, and denote $T(G)$.
	\item $\Z^t\cong G/T(G)$, so $t$ is the rank of $G/T(G)$.
	\item $k_r$ is the exponent of $T(G)$.
	\item Let $r$ be the smallest number of generator of $T(G)$, $T(G)=\Z/k_1\Z\oplus...\oplus \Z/k_r\Z$ can be generated by $r$ elements.
\end{enumerate}
Let $p|k_1$ be a prime. Then $T(G)/pT(G)=(\Z/p\Z)^r$. This is a vector space over $\Z/p\Z$. So cannot be spanned by fewer than $r$ elements.

As an exercise show that $k_i$ is the smallest positive integers so that $k_i\cdot T(G)$ can be generated by $r-i$ elements.

\begin{cor}
$k_i$ are unique for $G$, and called the invariant factors of $G$.	
\end{cor}
Show as an exercise that $r+t$ is the smallest number of generators of $G$.
\begin{definition}
Let $A$ be an abelian group. Then $T(A)$ is all the elements of of finite order in $A$. This is a subgroup of $A$.
\end{definition}
\begin{definition}
A subgroup $N$ of $G$ is characteristic if for every $\phi(N)=N$. 	
\end{definition}
As an exercise, show \begin{enumerate}
 \item $N$ is characteristic in $G$ implies that $N$ is normal in $G$.
 \item $T(A)$ is characteristic in $A$.	
 \end{enumerate}
\begin{definition}
$A$ is torsion if $A=T(A)$. $A$ is torsion free if $T(A)=\{0\}$.	
\end{definition}
\begin{prop}
$A/T(A)$ is torsion free.	
\end{prop}
\begin{definition}
Given $n\in\N$. Then $nA=\{na:a\in A\}\leq A$, and $A[n]=\{a\in A: na=0\}\leq A$.	
\end{definition}
Note that there is a natural injection from $A[n]$ into $A$, and a natural surjection from $A$ onto $nA$.
\begin{definition}
Let $P$ be a prime, then $A_p=\{a\in A:p^ka=0\text{ for some }k\in\N\}=\bigcup\limits_{k=	1}^\infty A[p^k]$. We call it the $p$-primary part of $A$.	
\end{definition}
Note that $A[p]\subseteq A[p^2]\subseteq...\subseteq A[p^n]\subseteq ...$
\begin{definition}
Let $H_i$ for $i\in I$ be a family of subgroups of $G$. It is a chain if for any $i,j\in I$, either $H_i\subseteq H_j$ or $H_j\subseteq H_i$.	
\end{definition}
Show as an exercise that the union of any chain of subgroups is a subgroup.
\begin{prop}
If $A$ is a torsion abelian group, then $A\cong \bigoplus\limits_{p\text{ prime }}A_p$
\end{prop}
\begin{proof}
Since $A_p$ are subgroups, we have the natural embeddings $A_p\hookrightarrow A$. Take the induced homomorphism $\bigoplus_pA_p\to A$. Then $(a_p)\mapsto \sum_pa_p$.

Let $a\in A$, and $n$ be the order of $a$. Then $n=p_1^{k_1}\cdots p_s^{k_s}$ is its prime factorization. Then $\frac{n}{p_i^{k_i}}a\in A_{p_i}$ since $p_i^{k_i}\cdot\frac{n}{p_1^{k_i}}a=na=0$.

We observe that $\frac{n}{p_1^{k_1}},...,\frac{n}{p_s^{k_s}}$ have non trivial common divisors, so $m_1\frac{n}{p_1^{k_1}}+...+m_s\frac{n}{p_s^{k_s}}=1$ for some $m_1,...,m_s$. So, $a=m_1\frac{n}{p_1^{k_1}}a+...+m_s\frac{n}{p_s^{k_s}}a$.

Suppose $a_{p_1}\in A_{p_i}$ and $a_{p_1}+...+a_{p_k}=0$. There is $N\st p_i^N\cdot a_{p_i}=0$ for all $p_i$. Then $p_2^N\cdot...\cdot p_t^N(a_{p_1}+\cdot +a_{p_t})=0=p_2^N\cdot...\cdotp_t^Na_{p_1}$, so order of $ap_1|p_2^N\cdot...\cdotp_t^N$ and so order of $ap_t|p_1^s$, therefore, s=0.
\end{proof}
\begin{note}
	$G$ a finite abelian group. Then $G=G_{p_1}\oplus...\oplus G_{p_s}$ for some $p_i$. Then $G_{p_i}\cong \Z/p_1^{m_{i1}}\Z\oplus...\oplus \Z/p_i^{m_{tk_i}}\Z$, $m_{i1}\leq ...\leq m_{ik_{i}}$.
	
	$G_{p_i}=p_i^{m_{i1}+...+m_{ik_{s}}}=p_i^{k_i}$ where $|G|=N=P_1^{k_1}\cdot ...\cdot p_s^{k_s}$.
\end{note}
\begin{cor}
	Every finite abelian group is a direct sum of cyclic groups of prime power orders and the collection of all prime power order is unique for the group. We call the prime powers appearing elementary divisors.
\end{cor}
\begin{example}
	$\Z/12\Z\oplus\Z/18\Z=\Z/3\Z\oplus\Z/4\Z\oplus\Z/2\Z\oplus\Z/9\Z=\Z/2\Z\oplus\Z/4\Z\oplus\Z/3\Z\oplus\Z/9\Z=\Z/6\Z\oplus\Z/36\Z$
\end{example}
\section{Feb. 16}
\begin{thm}
	$G$ finitely generated abelian group. Then 
	\begin{enumerate}
	\item $G\cong T(G)\times \Z^t$ for some $t$ which is unique and called the (torsion free) rank of $G$.
	\item $T(G)\cong \Z/k_1\Z\times...\times \Z/k_s\Z$ is finite, where $1<k_1|k_2|...|k_s$ are unique for $G$ and called the invariant factors of $G$.
	\item $T(G)\cong T(G)_{p_1}\times ...\times T(G)_{p_k}$ where $|T(G)|=p_1^{m_1}...p_k^{m_k}$, and the invariant factors of $T(G)_{p_i}$ together are unique for $G$ and called the elementary divisors of $G$.
	
	So, $T(G)$ is a direct sum of cyclic groups of prime power order in an essentially unique way.
	\end{enumerate}
\end{thm}
\begin{definition}
	$G$ abelian group, $n\in N$. Then
	\begin{enumerate}
		\item $G[n]=\{g\in G: ng=0\}$ is a subgroup.
		\item $nG=\{ng:g\in G\}$ is a subgroup.
		\item $p$ a prime. $G_p=\{g\in G: p^kg=o\text{ for some }k\}=\bigcup_kG[p^k]$ is a subgroup called the $p$-primary component.
		\item $T[G]=\{g:ng=0\text{ for some } n>0\}=\bigcup_nG[n!]$ is a subgroup.
	\end{enumerate}
\end{definition}
Note, we have $G/T(G)$ is torsion-free.
\begin{thm}
	If $G$ torsion, then $G\cong \bigoplus\limits_{p\text{ prime}}G_p$.
\end{thm}
Show as an exercise that if $G$ is abelian and $G/A$ is free abelian, then $G\cong A\times G/A$.

Warning: $T(G)$ is not always a direct summand to $G$ ($G\not\cong T(G)\times G/T(G)$)
\begin{example}
Consider $(\Q,+)$. Every $2$ elements of $\Q$ are dependent, for $\frac{p}{q}$, $\frac{m}{n}$, we have $mq\frac{p}{q}-pn\frac{m}{n}=0$. So, $\Q$ is not free abelian, it is torsion-free, not cyclic.
\end{example}
\begin{example}
Consider $S^1=\{z\in\C:|z|=1\}$ with $\cdot$.

$T(S^1)=\mu_\infty$ is all roots of unity which is $\{e^{2\pi i\frac{m}{n}}: \frac{m}{n}\in\Q\}$.

$T(S^1)_p=\mu_p^\infty$ is all roots of unity of $p$-power order.

We have a surjective homomorphis, $E:(\R,+)\to S^1$ by $t\mapsto e^{2\pi i t}=\cos(2\pi t)+i\sin(2\pi t)$. Here, $\ker E=\Z$. So, $S^1\cong \R/\Z$ with $E^{-1}(T(S^1))=\Q$.

So, $\mu_\infty\cong \Q/\Z$ and $\mu_p^\infty=\{$rational numbers with $p$th power denominators$\}/\Z$

$S^1/T(S^1)\cong (\R/\Z)/(\Q/\Z)\cong \R/\Q\cong \bigoplus \Z$.	
\end{example}

As an exercise. show that $S^1\cong T(S^1)\times \R/\Q$ where $\R/\Q\cong S^1/T(S^1)$.

Note that $\mu_p^\infty$ is infinite but every proper subgroup is finite and cyclic.
\begin{definition}
	$G$ abelian, $n\in\N$. Then
	\begin{enumerate}
	\item $a\in G$ is $n$-divisible if $a=nb$ for some $b\in G$.
	\item $G$ is $n$-divisible if all elements of $G$ are $n$-divisible.
	\item $G$ is divisible if it is $n$-divisible for every $n$.	
	\end{enumerate}
\end{definition}
\begin{example}
	$\Q$ is divisible, $\Q/\Z$ is divisible, $\mu_p^\infty$ are divisible, $S^1$ is divisible.	
\end{example}
Show as an exercise that if $G$ is divisible then $G/A$ is divisible for any $A\leq G$. Also, if $A$ is divisible then $A\leq G\implies G\cong A\oplus G/A$.
\begin{definition}
$G$ is abelian. $A\leq G$, then $A$ is called \emph{pure} in $G$ if for any $a\in A$ and any $n\in \Z$ if $a=ng$ for some 	$G\in G$ then $a=nb$ for some $b\in A$ (i.e., $A\cap nG=nA$).
\end{definition}
\begin{thm}
Every divisible group is a direct sum of groups isomorphic to $\Q$ or $\mu_p^\infty$ for some prime $p$.
\end{thm}
\begin{note}
$A$ torsion and $A[n]=\{0\}$ then $A=nA$. If $|g|=k$, $\gcd(n,k)=1$ then $\langle g^n\rangle=\langle g\rangle$.	
\end{note}
\begin{thm}
	$G$ abelian, $A<G$ pure, $G/A$ a direct sum of cyclic groups (i.e., $G/A$ has a basis), then $G\cong A\oplus G/A$
\end{thm}
\begin{thm}
$G=G_p$ is an abelian $p$-group of finite exponent ($G=G[p^k]$ for some $k$) then $G$ is a direct sum of cyclic groups.	
\end{thm}
\begin{cor}
If $G$ abelian of finite exponent, then $G$ is a direct sum of cyclic groups.
\end{cor}
Show as an exercise that $T(G)$ is always pure in $G$.
\begin{thm}
If $T(G)$ is of finite exponent then $G\cong T(G)\times G/T(G)$.	
\end{thm}
\section{Feb. 18}
\begin{thm}
An abelian group of finite exponent is a direct sum of cyclic groups.	
\end{thm}
\begin{thm}
	If $A\leq G$ and $A$ is pure in $G$ and $G/A$ is a direct sum of cyclic group, then $G\cong A\times G/A$. 	
\end{thm}
\begin{thm}
$A\leq G$ pure and of finite exponent, then $G\cong A\oplus G/A$.	
\end{thm}
\begin{thm}
If $T(A)$ is of finite exponent then $A\cong T(A)\times A/T(A)$.	
\end{thm}
\begin{thm}[Kulikov]
	$G$ torsion abelian then $G$ has a pure subgroup $A$ which is a direct sum of cyclic groups and $G/A$ is divisible.
	$$A\hookrightarrow G\twoheadrightarrow G/A$$
\end{thm}
Let $G$ be a group. $X\subseteq G\st G=\langle X\rangle$. This means that every elememnt of $G$ is of the form $g_1^{\epsilon_1}...g_k^{\epsilon_k}$ with $g_i\in X$, $\epsilon_i=\pm 1$.

Usually there are many ways a given element can be written like.

Trivial reasons: We can always insert somewhere $gg^{-1}$ or $g^{-1}g$; $g\in X$.

Question: Are there groups $G$ and $X\subseteq G$ where this is the only reason?

\begin{definition}
	$X$ a set. A word of length $n$ over $X$ is a sequence of $n$ elements from $X$ (repetition allowed): $a_1a_2...a_n$ where $a_i\in X$. Note, word of length 0 is the empty word.
	
	$W(X)$ is the set of all finite words. Given $2$ words, $u,w\in W(X)$, we can concatenate them with $u\star w=uw$. This is an associative binary operation, and it makes $W(X)$ a monoid. It is called the free monoid on $X$. 
\end{definition}
Show as an exercise that given any monoid $M$ and any function $f:X\to M$ it extends uniquely to a homomorphism $W(X)\to M$.

\begin{definition}
	$X$ a set. Consider $X\times \{1,-1\}$. We write $x$  for $(x,1)$ and $x^{-1}$ for $(x,-1)$. Consider $W(X\times \{1,-1\})$.
	
	A word $x_1^{\epsilon_1}x_2^{\epsilon_2}...x_n^{\epsilon_n}$ in $W(X\times \{1,-1\})$ is \emph{reduced} if whenever $x_i=x_{i+1}$ we have $\epsilon_i\not=-\epsilon_{i+1}$.
	
	$R(X)$ is the set of all reduced words in $W(X\times \{1,-1\})$.
\end{definition}
\begin{note}
	$M$ is a groupa and $f:X\to M$ then it extends to $f:X\times \{-1,1\}\to M$ by $(x,1)\mapsto f(x)$ and $(x,-1)\mapsto f(x)^{-1}$ and it extends to monoid homomorphism $W(X\times \{1,-1\})\to M$. Clearly equivalent words have the same images in $M$. 

$R(X)$ is the set of reduced words in $W(X\times \{1,-1\})$ and it has a binary operation $u\star w=uw$ and reduced.

This operation has inverses as $(x_1^{\epsilon_1}...x_n^{\epsilon_n})^{-1}=x_n^{-\epsilon_n}...x_1^{-\epsilon_1}$. We have $(x_1^{\epsilon_1}...x_n^{\epsilon_n}) (x_n^{-\epsilon_n}...x_1^{-\epsilon_1})=\emptyset$	
\end{note}

Problem is that is this operation associative? Yes, but technical complication.

\begin{definition}
	$G$ a group. $X\subseteq G$ a subset. We say $X$ generates $G$ freely if the natural map $R(X)\twoheadrightarrow G$ is bijective (So, $X$ generates $G$).
	
	If this happens than $R(X)$ is a group.
\end{definition}
Note that if $X$ generates freely $G$, $Y$ generates
freely $H$. $f:X\to Y$ is a bijection, then it extends to an isomorphism $G\to H$.
\begin{example}
Let $X=\{1\}$, we have $G=\Z$ and $\{1\}$ generates freely $\Z$.	
\end{example}
Show as an exercise that if $X$ generates freely $G$, $f:X\to H$ any function to a group $H$, then it extends uniquely to a homomorphism $G\to H$.

\section{Feb. 21}
\begin{definition}
	$X$ a set. $W(X\times\{1,-1\})$ is the free monoid. Then $R(X)$ is all reduced words in $X\cup X^{-1}$ which is a subgroup of $W(X\times \{1,-1\})$. $R(X)$ has a binary operation with every element ``invertible,'' but not yet established that it is surjective.
\end{definition}
Given any group $G$ and a function $X:X\to G$, there is a unique monoid homomorphism $f:W(X\times\{1,-1\})\to G$ by $x\mapsto f(x)$ and $x^{-1}\mapsto f(x^{-1})$ for $x\in X$ and it restricts to a ``homomorphism'' on $R(X)$. 

\begin{definition}
	Let $G$ be a group with generating set $X$. We say that $X$ \emph{generates freely} $G$ if the natural map $R(X)\to G$ is a bijection.
	
	If such a group exists, then $R(X)$ is a group.
\end{definition}
\begin{note}
	If $R(X)$ is not a bijection, then there is a non trivial reduced word $w$ which is mapped onto $e\in G$.	
\end{note}
\begin{proof}
	Choose shortest reduced word $u$ s.t. $f(u)=f(v)$ for some $v\not=u$. If $u=\emptyset$, then $w=v$ works.
	
	Otherwise, suppose $u$ starts with $x^\epsilon$, $x\in X$, $\epsilon=\pm 1$ and $u=x^\epsilon u_1$. If $v=x^\epsilon v_1$, then $f(u)=f(x)^\epsilon f(u_1)=f(x)^\epsilon f(v_j)$. So $f(u_1)=f(v_1)$ and $u_1$ is shorter which is a contradiction. So, $v\not=x^{\epsilon}v_1$ and therefore $u^{-1}v$ is reduced and $f(u^{-1}v)=f(u)^{-1}f(v)=e$. So $G$ is freely generated by $X$ iff $G$ is generated by $X$ and no non-trivial reduced word in $X$ represents $e$.
 \end{proof}
\begin{definition}
	Assume free group on $2$ elements exists, $G=\langle a,b\rangle$ is freely generated by $a,b$.
	
	Notation, for $x$ a letter, $n\in\Z_{\not=0}$ define $X^n=\begin{cases}
		\overbrace{x\cdot ...\cdot x}^n&n>0\\
		\underbrace{x^{-1}\cdot ...\cdot x^{-1}}_{-n}&n<0
	\end{cases}$
\end{definition}
\begin{note}
	Reduced	words in $a,b$ are of the form $a^{n_1}b^{n_2}...c^{n_k}$ where $c=a$ or $b$, or $b^{n_1}a^{n2}...c^{n_k}$ where $c=a$ or $b$.
\end{note}
\begin{thm}
	Let $a=\begin{bmatrix}
		1&2\\0&1
	\end{bmatrix}$, $b=\begin{bmatrix}
1&0\\2&1	
\end{bmatrix}\in SL_2(\Z)
$. The subgroup $\langle a,b\rangle$ of $SL_2(\Z)$ is freely generated by $\{a,b\}$.
\end{thm}
\begin{proof}
	Let $w$ be a non-trivial reduced word in $F(\{a,b\})$. We need to show $w\not=e$ in $\langle a,b\rangle$.

First, assume that $w$ starts with $b$ or $b^{-1}$; i.e., $w=b^i...c^{\epsilon}$ where $i,\epsilon\in\{1,-1\}$ and $c\in\{a,b\}$. Take $\delta=\begin{cases}
	1&\text{if }c^\epsilon=a,b,b^{-1}\\
	-1&\text{if} c^\epsilon=a^{-1}
\end{cases}$, and $u=a^{-\delta}wa^\delta$. Since $a^{-\delta}$ and $a^{\delta}$ does not cancel with $b^i$ and $c^\epsilon$ respectively, $u$ is also a reduced word. If $w=e$, then $u=a^{-\delta}ea^\delta=e$; and if $u=e$, then $w=a^\delta ea^{-\delta}$. So, $w=e$ iff. $u=e$. So, it suffices to show that $w=a^{d_1}b^{d_2}...c^{d_k}$ where $c\in\{a,b\}$, $d_1,...,d_k\in\Z_{\not=0}$ is not $e$.

We will first show by induction that $a^d=\begin{bmatrix}
	1&dz\\0&1
\end{bmatrix}$ and $b^d=\begin{bmatrix}
	1&0\\dz&1
\end{bmatrix}$ for $d\in\Z$. By definition, $a^0=\begin{bmatrix}
	1&0\\0&1
\end{bmatrix}$ and $b^0=\begin{bmatrix}
	1&0\\0&1
\end{bmatrix}$. If $a^d=\begin{bmatrix}
	1&dz\\0&1
\end{bmatrix}$ and $b^d=\begin{bmatrix}
	1&0\\dz&1
\end{bmatrix}$, then $a^{d+1}=\begin{bmatrix}
	1&dz\\0&1
\end{bmatrix}\begin{bmatrix}
	1&z\\0&1
\end{bmatrix}=\begin{bmatrix}
	1&(d+1)z\\0&1
\end{bmatrix}$ and similarly, $b^{d+1}=\begin{bmatrix}
	1&0\\dz&1
\end{bmatrix}\begin{bmatrix}
	1&0\\z&1
\end{bmatrix}=\begin{bmatrix}
	1&0\\(d+1)z&1
\end{bmatrix}$. So, by PMI, this is true for $d\in\N$. Now, since $\begin{bmatrix}
	1&dz\\0&1
\end{bmatrix}\begin{bmatrix}
	1&-dz\\0&1
\end{bmatrix}=\begin{bmatrix}
	1&0\\0&1
\end{bmatrix}$, we have $a^{-dz}=\begin{bmatrix}
	1&-dz\\0&1
\end{bmatrix}$; and similarly, we have $b^{-dz}=\begin{bmatrix}
	1&0\\-dz&1
\end{bmatrix}$. Therefore, $\forall d\in\Z$, $a^d=\begin{bmatrix}
	1&dz\\0&1
\end{bmatrix}$ and $b^d=\begin{bmatrix}
	1&0\\dz&1
\end{bmatrix}$.

Now, define $(\alpha_i)$ recursively by $\alpha_0=1$, $\alpha_1=d_1z$, and for $n\geq 2$, $\alpha_n=\alpha_{n-2}+d_nz\alpha_{n-1}$ where $d_n$ are such powers that are defined in $w=a^{d_1}b^{d_2}...c^{d_k}$. We will now induct on $k$ to show that $w=\begin{cases}
	\begin{bmatrix}
		\alpha_k&\alpha_{k-1}\\\cdot&\cdot
	\end{bmatrix}&\text{if $k$ is even}\\
	\begin{bmatrix}
		\alpha_{k-1}&\alpha_k\\\cdot&\cdot
	\end{bmatrix}&\text{if $k$ is odd}
\end{cases}$ for $k\in\N$.

If $k=1$, then $w=a^{d_1}=\begin{bmatrix}
	1&d_1z\\0&1
\end{bmatrix}=\begin{bmatrix}
	\alpha_0&\alpha_1\\\cdot&\cdot
\end{bmatrix}$.

If $k=2$, then $w=a^{d_1}b^{d_2}=\begin{bmatrix}
	\alpha_0&\alpha_1\\\cdot&\cdot
\end{bmatrix}\begin{bmatrix}
	1&0\\d_2z&1
\end{bmatrix}=\begin{bmatrix}
	\alpha_0+\alpha_1d_2z&\alpha_1\\\cdot&\cdot
\end{bmatrix}=\begin{bmatrix}
	\alpha_2&\alpha_1\\\cdot&\cdot
\end{bmatrix}$.

Now, assume for some odd $k>2$, we have $a^{d_1}b^{d_2}...b^{k-1}=\begin{bmatrix}
	\alpha_{k-1}&\alpha_{k-2}\\\cdot&\cdot
\end{bmatrix}$. Then $a^{d_1}b^{d_2}...b^{d_{k-1}}a^{d_k}=\begin{bmatrix}
	\alpha_{k-1}&\alpha_{k-2}\\\cdot&\cdot
\end{bmatrix}\begin{bmatrix}
	1&d_kz\\0&1
\end{bmatrix}=\begin{bmatrix}
	\alpha_{k-1}&\alpha_{k-2}+d_kz\alpha_{k-1}\\\cdot&\cdot
\end{bmatrix}=\begin{bmatrix}
	\alpha_{k-1}&\alpha_k\\\cdot&\cdot
\end{bmatrix}$.

Similarly, assume for some even $k>2$, we have $a^{d_1}b^{d_2}...a^{k-1}=\begin{bmatrix}
	\alpha_{k-2}&\alpha_{d_{k-1}}\\\cdot&\cdot
\end{bmatrix}$ Then $a^{d_1}b^{d_2}...a^{k-1}b^k=\begin{bmatrix}
	\alpha_{k-2}&\alpha_{k-1}\\\cdot&\cdot
\end{bmatrix}\begin{bmatrix}
	1&0\\d_kz&1
\end{bmatrix}=\begin{bmatrix}
	\alpha_{k-2}+d_kz\alpha_{k-1}&\alpha_{k-1}\\\cdot&\cdot
\end{bmatrix}=\begin{bmatrix}
	\alpha_{k}&\alpha_{k-1}\\\cdot&\cdot
\end{bmatrix}$.

Therefore, by PMI, $w=\begin{cases}
	\begin{bmatrix}
		\alpha_k&\alpha_{k-1}\\\cdot&\cdot
	\end{bmatrix}&\text{if $k$ is even}\\
	\begin{bmatrix}
		\alpha_{k-1}&\alpha_k\\\cdot&\cdot
	\end{bmatrix}&\text{if $k$ is odd}
\end{cases}$.

Consider $|\alpha_i|$, we will show that $|\alpha_i|$ is an increasing sequence and thus never $=0$.

Since $|z|\geq 2$, $\alpha_1=|d_1z|=|d_1||z|\geq 2>|\alpha_0|$ as $d_1\not=0$.  If $|\alpha_{k-1}|>|\alpha_{k-2}|$, then $|\alpha_k|=|\alpha_{k-2}+d_kz\alpha_{k-1}>|d_kz||\alpha_{k-1}|-|\alpha_{k-2}|>(|d_kz|-1)|\alpha_{k-1}|>(2-1)|\alpha_{k-1}|=|\alpha_{k-1}|$. Therefore, $|\alpha_i|$ is an increasing sequence by PMI. So, $\forall k$, $|a_k|\not=0$ and thus $w\not=e$.

Therefore, $\langle a,b\rangle$ is free.
\end{proof}
\begin{prop}
	Let $x_n=a^nba^n$ where $n=1,2,3,...$. Then $H=\langle x_1,x_2,...\rangle$ is freely generated by $x_1, x_2,...$. 
\end{prop}
\begin{proof}
	$x_n^{-1}$ is represented in $G$ by $a^{-n}b^{-1}a^{-n}$. Elements of $X\cup X^{-1}$ are of the form, $a^mb^{\epsilon_m}a^m$ where $m\in\Z$ and $m\not=0$. Now reduced words in $R(x_1,...)$ look like $a^{m_1}b^{\epsilon_1}a^{m_2}b^{\epsilon_2}a^{m_2}...a^{m_k}b^{\epsilon_k}a^{m_k}$; $\epsilon_i=$ sign $m_i$ and $m_i+m_{i+1}\not=0$. So these are also non-trivial reduced words of $a,b$ and hence non-zero.
\end{proof}
\begin{cor}
For any finite set $X$, $R(X)$ is a group (i.e., the operation is associative).	
\end{cor}
\begin{cor}
For every $X$, $R(X)$ is a group.	
\end{cor}
\begin{proof}
	Take 3 reduced words, $u,v,w$. We need $(uv)w=u(vw)$. But $u,v,w\in R(Y)$ for some finite subset $Y$ of $X$ which we know is a group.
\end{proof}
\begin{definition}
	$A$ a group. It is \emph{free} if it is freely generate by a subset $X$. Then $A=R(X)=\Free(X)$.
\end{definition}
\begin{thm}
	Every group is isomorphic to a quotient of a free group.
\end{thm}
\begin{proof}
	We have a surjective homomorphism $\Free(G)\twoheadrightarrow G$, so $G\cong \Free(G)/\ker$.
\end{proof}
\begin{definition}
	Let $(w_i)_{i\in I}$ be words of $\Free(X)$. Let $H$ be the smallest normal subgroup of $\Free(X)$ generated by $\{w_i:i\in I\}$. Then $\langle X|w_i,i\in I\rangle$ is the group $\Free(X)/H$.
\end{definition}
\begin{example}
	$\langle \{a\}|a^n\}=\Z/n\Z$, for $n>0$
\end{example}
\begin{thm}
	A subgroup of a free group is free.
\end{thm}
\section{Feb. 23}
\begin{thm}
	Let $a=\begin{bmatrix}
		1&2\\0&1
	\end{bmatrix}$, $b=\begin{bmatrix}
		1&0\\2&1
	\end{bmatrix}\in \SL_2(\Z)$ and $H=\langle a,b\rangle$. Then this is freely generated by $\{a,b\}$.
\end{thm}
\begin{cor}
For any set $X$, the structure $R(X)$ is a group, denoted $\Free(X)$ and called the free group on $X$.	
\end{cor}
\begin{thm}
	Any group is isomorphic to a quotient of a free group.
\end{thm}
\begin{definition}
	Given a set $X$ and a collection of reduced words $w_i$, $i\in I$ in $\Free(X)$. Then $\langle X| w_i,i\in I\rangle =\Free(X)/N$ with $N$ is the smallest normal subgroups of $\Free(X)$ which contains all $w_i,i\in I$. If a group $G$ is isomorphic to $\langle X|w_i,i\in I\rangle$. Then any isomorphism $\langle X|w_i,i\in I\rangle\to G$ is called a presentation of $G$.
\end{definition}
\begin{example}
	$D_\infty=\rangle a,b|a^{2},b^2\langle=\rangle c,d|d^2,dcd^{-1}c\rangle$.
\end{example}
\begin{definition}
	$G$ is called finitely presented if it has a presentation of finitely generators and finitely many relations.
\end{definition}
\begin{thm}
	Any finite group is finitely presented.
\end{thm}
\begin{proof}
	$G$ finite. $G\cong \Free(X)/N$ where $X$ is finite. So $N$ is of finite index in $\Free(X)$.
\end{proof}
\begin{thm}
	A subgroup of finite index in a finitely generated group is finitely generated. 
\end{thm}
\begin{example}
	$\Z^2\cong\langle a,b|a^{-1}b^{-1}ab\rangle =\Free(\{a,b\})/N$ but $N$ is not finitely generated as $N=[\Free(a,b),\Free(a,b)]$
 \end{example}
Goal: To prove the Nielsen-Schreier theorme. A subgroup of any free group is free.

$G$ a group. $X$ a generating set, $H\leq G$. Let $S$ be the set of choice of left coset representatives for $H$ in $G\st e\in S$.

For any $g\in G$, there is a unique $\bar g\in S\st gH=\bar gH$.

\begin{note}
	$\bar{(\bar{g})}=\bar g$, $\bar{g_1g_2}=\bar{g_1\bar{g_2}}$. For $s\in S$, $\bar s=s$ and $\forall g$, $\bar g^{-1}g\in H$ and for all $h\in H$, $\bar{h}=e$,

Given $g\in G$, $s\in S$, there is unique $t\in S\st t^{-1}gs\in H$ with $t=\bar{gs}$. We denote $t^{-1}gs$ by $h(g,s)=(\bar{gs})^{-1}(gs)$; i.e., $h(g,s)=t^{-1}gs$.

Here, $h(g,s)^{-1}=s^{-1}g^{-1}t=h(g^{-1},t)$.
\end{note}
\begin{prop}
	Let $Y=\{h(x,s):x\in X,s\in S\}$, then $Y'=\{h(x^{-1},s):x\in X,s\in S\}$. Thus, $H=\langle Y\rangle$.
\end{prop}
\begin{definition}
	Let $G=\Free(X)$, $H\leq G$. A set $S$ is called a Schreier set for $H$ if it is a set of left coset representatives for $H$ in $G$ and if a reduced word $x_1^{\epsilon_1}\mu\in S$, then also $\mu\in S$. (with any reduced word in $S$, all its final sequences are in $S$). 
\end{definition}
\section{Feb. 25}
\begin{thm}[Nielson-Schrier]
A subgroup of a free group is free.
\end{thm}
\begin{proof}[Outline of Proof]
	$G=\Free(X)$ a free group. $H\leq G$. $S$ a Schrier set for $H$ ($S$ is a left coset representative for $H\st$ if a reduced word is in $S$, then all its final segments are in $S$). 
	
	Given $x\in X$, $s\in S$, there is one unique $t\in S\st h(x,s)\in H\st t^{-1}xs=h(xs)$. Let $Y=\{h(x,s):x\in X,s\in S,h(x,s)\not=e\}$ We look at a reduced word $h(x_1,a_1)^{\epsilon_1}...h(x_n,a_n)^{\epsilon_n}=$
	$$t_1^{-1}x_1^{\epsilon_1}s_1t_2^{-1}x_2^{\epsilon_2}s_2...t_n^{-1}x_n^{\epsilon_n}s_n$$
	
	Here, each $t_i^{-1}x_i^{\epsilon_i}s_i$ is a reduced word and study possible collections in $t_i^{-1}x_i^{\epsilon_i}s_i$ show that all the letters $x_i^{\epsilon_i}$ will survive, so this element is not $\emptyset$.
	
	Note that $h(x,s)=e\iff xs\in S$.
\end{proof}
	As an exercise, show that if $|X|=k$ and $|S|=[G:H]<\infty$, then $h(x,s)=e$ for exactly $[G:H]-1$ pairs $(x,s)$, so $|Y|=k[G:H]-[G:H]+1=(k-1)[G:H]+1$.

\begin{thm}
	A subgroup of index $n$ in a free group of rank  $k$ is free of rank $(k-1)n+1$.
\end{thm}
As an exercise, find a Schrerier set for the commutator subgroup of $\Free(\{a,b\})$. Show that if $N\unlhd \Free(X)$ and $[\Free(X):N]=\infty$ and $N\not=\{e\}$, then $N$ is not finitely generated. Also, let $F_1,F_2$ be free subgroups of $G$ s.t. $[G:F_1]=[G:F_2]<\infty$, show that they have the same rank.
\begin{definition}
	$G$ a group. 
	\begin{enumerate}
		\item The \emph{center} of $G$, $Z(G)=\{a\in G:[g,a]=e\text{ for all }g\in G\}$. Note that $[h,g]=hgh^{-1}g^{-1}$.
		We always have $Z(G)\unlhd G$.
		\item $[G,G]=G'=\langle \{[h,g]:h,g\in G\}\rangle$ is the \emph{derived group} of $G$, also called the \emph{commutator subgroup} of $G$.
	\end{enumerate}
\end{definition}
\begin{thm}
	Let $f:G\to A$ be a homomorphism to an abelian group. Then $f([h,g])=e$ for any $h,g\in G$.
\end{thm}
\begin{thm}
	$[G,G]$ is normal in $G$.
\end{thm}
\begin{proof}
	We have $a[h,g]a^{-1}=[aha^{-1},aga^{-1}]\in[G,g]$.
\end{proof}
\begin{definition}
	The \emph{abelianization} of $G$ is $G^{ab}=G/[G,G]$. This is an abelian group.
\end{definition}
\begin{cor}
	$G$ a group, $A$ abelian, $f$ a homomorphism as shown in this diagram. \begin{tikzcd}
	G\arrow[r,"f"]\arrow[two heads, d,"\pi"']&A\\
	G^{ab}\arrow[dotted, ur]	
	\end{tikzcd}
Then, $[G,G]\subseteq \ker f$.
\end{cor}
\begin{definition}
	$G$ is \emph{perfect} if $G=[G:G]$;i.e., $G^{ab}=\{e\}$.
\end{definition}
\begin{cor}
	$G$ is simple (has only trivial normal proper subgroup) iff $G$ is abelian or perfect.
\end{cor}
\begin{definition}
	$A,b$ subsets of $G$. Then $[a,b]=\langle[a,b]:a\in A, b\in B\rangle$.
\end{definition}
\begin{prop}
	If $G=KH$ where $K\unlhd G$ and $H\unlhd G$ with $K\cap H=\{e\}$, then $G\cong K\times H$.
	
	In particular, $[K,H]=\{e\}$
\end{prop}
We will now study when $G=KH$, $K\unlhd G$ and $K\cap H=\{e\}$ with no assumptions about the normality of $H$.

\begin{note}
If $K\unlhd G$, we get a homomorphism $G\to \Aut(K)$ with $g\mapsto C_g:h\mapsto ghg^{-1}$.

The kernel is denoted the \emph{centralizer} of $K$ in $G$, $C_G(k)$.
\end{note}

Now, restricting this to $H$, we get $\phi: H\to \Aut(K)$. Since $G=KH$ and $K\cap H=\{e\}$, we have every $g\in G$ is uniquely expressed as $g=k\cdot h$ where $k\in K$ and $h\in H$ since  $kh=k_1h_1$ implies thta $kk_1^{-1}=h_1h^{-1}=e$.

Hence, we get a bijection $G\to K\times H$ where $(kh)(k_1h_1)=k(hk_1h^{-1})=kC_h(k_1)h$.

\begin{definition}
	Given that $K,H$ groups, homomorphism $\phi:H\to \Aut(K)$, define the \emph{semidirect prodcut} of $H$ by $K$, $K\rtimes_\phi H=K\times H$ with $(k,h)\star (k_1,h_1)=(k\phi_{h}(k_1),hh_1)$.
\end{definition}
As an exercise, show that this is a group operation on $K\times H$.
\begin{note}
	$K\cong K\times\{e\}$, $H\cong \{e\}\times H$, and $hkh^{-1}=\phi_{h}(k)$.
\end{note}

\begin{example}
	$A$ a cyclic group ($A\cong\Z/n\Z$ or $A\cong \Z$), then $A$ always has the following automorphism.
	\begin{enumerate}
		\item $\id: A\to A$.
		\item $\phi:a\mapsto a^{-1}$
	\end{enumerate}
	
	We note that $\{\id, \phi\}\cong \Z/2\Z$ and if we take $\eta:\Z/2\Z\to \Aut(A)$, we have $A\rtimes_\eta \Z/2\Z$ is a dihedral group. 
\end{example}

\section{Feb. 28}
\begin{definition}
	$K,H$ groups, $\phi:H\to \Aut(K)$ a homomorphism (denote $\phi_k=\phi(k)$). Then $K\rtimes_\phi H=K\times H$ as a set, with $(k,h)\cdot(k_1,h_1)=(k\phi_h(k_1),hh_1)$.
\end{definition}
As an exercise, show that this is a group structure on $K\rtimes_\phi H$ which is called the \emph{semi-direct} product of $H$ by $K$, we correspond $(k,0)$ with $K$ and $(0,h)$ with $H$.

\begin{thm}
	We have $K\unlhd K\rtimes_\phi H$, $H\leq K\rtimes_\phi H$, $K\cap H=\{e\}$, $K\rtimes_\phi H=KH$ and $hkh^{-1}=\phi_h(k)$.
	
	Conversely, if $K\unlhd G$, $H\leq G$, $K\cap H=\{e\}$, $G=KH$, then $G\cong K\rtimes_\phi H$ where $\phi:H\to\Aut(K)$ by $h\mapsto C_h$.
\end{thm}
\begin{example}
	$A$ abelian. Then $\Aut(A)$ contains $id$ and $\eta:a\mapsto a^{-1}$. So we have a homomorphism $\phi:\Z/2\Z\to \Aut A$by $0\mapsto \id$ and $1\mapsto \eta$. We thus construct $A\rtimes_\phi \Z/2\Z$.
	
	In particular, if $A=\Z/n\Z$, we have $\Z/n\Z\rtimes \Z/2\Z\cong D_n$ and if $A=\Z$, $\Z\rtimes \Z/2\Z\cong D_\infty$.
\end{example}
\begin{example}
	Take $N=\Z/p\Z\times\Z/p\Z$ where $p$ is a prime. Then $\Aut(N)=\GL_2(\Z/p\Z)$.
	
	Take $\eta:N\to N$ by $\eta(a,b)=(a,a+b)$; i.e., $\eta=\begin{bmatrix}
		1&1\\0&1
	\end{bmatrix}$
	
	Note that $\eta^k=\begin{bmatrix}
		1&k\\0&1
	\end{bmatrix}$, so that $\eta^p=\id$.
	
	We thus have a homomorphism $\phi:\Z/p\Z\to \Aut(N)$ by $1\mapsto \eta$ and then , we have $P=N\rtimes_\phi\Z/p\Z$.
	
	Note that $|P|=p^3$, $\exp(P)=p$, and $P$ is non-abelian. 
\end{example}
 Show as an exercise that $P\cong \langle a,b,c\ |\ a^p,b^p,c^p,cbc^{-1}a^{-1}b^{-1},[a,b],[a,c]\rangle$.
 \begin{example}
 	Let $N=\Z/p^2\Z$. The map $\eta:N\to N$ by $a\mapsto (1+p)a$ is an automorphism of order $p$ since $\eta^p(a)=(1+p)^pa$ where $(1+p)^a=1+\binom{p}{1}p+\binom{p}{2}P^2+...=a+p^A\equiv 1(\mod p^2)$, so $\eta^p=\id$.
 	
 	We get a homomorphism $\phi:\Z/p\Z\to \Aut(\Z/p^{2}\Z)$ by $1\mapsto \eta$ and get $Q=(\Z/p^{2}\Z)\rtimes_\phi \Z/p\Z$.
 	
 	Note that $|Q|=p^3$, $\exp(Q)=p^2$, and $Q$ is non-abelian.
 \end{example}
 
 Show as an exercise that $Q\cong\langle a,b\ |\ a^{p^2}, b^{p}, (bab^{-1}a^{-1})^{-p}$.
 
 \begin{thm}
 	If $p$ is an odd prime, then
 	\begin{enumerate}
 		\item Every group of order $p$ is cyclic.
 		\item Every group of order $p^2$ is abelian (either $\Z/p^2\Z$ or $\Z/p\Z\times \Z/p\Z$).
 		\item A non abelian group of order $p^3$ is isomorphic to either $P$ or $Q$.
 	\end{enumerate}
 	
 	Also, we have
 	\begin{enumerate}
 		\item Every group of exponent $2$ is abelian.
 		\item Every group of order $4$ is abelian.
 		\item A non-abelian group of order $8$ is isomorphic to $D_4$ or $Q_8$.
 	\end{enumerate}
 \end{thm}
 \begin{example}
 	$R$ a commutative ring. $R^n=N$, $\Aut(R^n)\supseteq \GL_N(R)$.
 	
 	Then $\Aff(n,R)=\{f:R^n\to R^n:f(v)=Av+w\text{ for all $v\in R^n$ and some $A\in\GL_n(R), w\in R^n$}\}$.
 	
 	We have $\Aff(n,R)\cong R^n\rtimes \GL_n(R)$.
 \end{example}
Given a group $G$, we want to understand $\Aut(G)$.
\begin{definition}
	$K\leq G$ is called \emph{characteristic} if $\phi(K)=K$ for all $\phi\in\Aut(G)$
\end{definition}
As an exercise show that if $K$ is characteristic, the it is normal in $G$.
\begin{example}
	First we have that $Z(G)$ and $[G,G]$ are characteristic in $G$.
	
	If $A$ abelian, then $A[n]$, $nA$ are characteristic in $A$ for all $n\in \N$. The $p$-primary component $A_p$ is also characteristic for all $p$ prime. Therefore, $T(A)$ is characteristic.
\end{example}
Recall that $\Inn(G)$ is all inner automorphism of $G$ $\subseteq \Aut(G)$. $\Inn(G)\cong G/Z(G)$.
Show as an exercise that $Inn(G)\unlhd \Aut(G)$, so we can define the outer automorphisms of $G$, $\Out(G)=\Aut(G)/\Inn(G)$.
\begin{definition}
	$G$ is called \emph{complete} if $Z(G)=\{1\}$ and $\Aut(G)=\Inn(G)=G$.
\end{definition}
\begin{example}
	$\Aut(\Z)=\Z/2\Z$.
	
	Consider $\Aut(D_\infty)$.
	
	Recall, $D_\infty=\langle T,S\ |\ S^2=e, STST \rangle$. Given any group $G$ with $a,b\st b^2=e$ and $baba=e$, there is a unique homomorphism $D_\infty\to G$ by $T\mapsto a$ and $S\mapsto b$.
	
	So, we have $D_\infty\to D_\infty$ by $T\mapsto T^\epsilon$ and $S\to ST^2$, to be surjective, $\epsilon=\pm 1$.
	
	For every $\epsilon, L$, there is one such automorphism.
	
	Take $\alpha:D_\infty\to D_\infty$ by $\alpha(T)=T^{-1}$ and $\alpha(S)=S$ and $\beta:D_\infty\to D_\infty$ by $\beta(T)=T$ and $\beta(S)=ST$. Then, $\Aut(D_\infty)=\langle \alpha,\beta\rangle\cong D_\infty$.
	
	We note that $Z(D_\infty)=\{1\}$ and $D_\infty\cong \Inn(D_\infty)\subset \Aut(D_\infty)$ and $\Out(D_\infty)=\Z/2\Z$.
\end{example}
Show as an exercise that $\Aut(D_n)\cong \Aff(1,\Z/n\Z)\cong (\Z/n\Z)\rtimes(\Z/n\Z)^\times$. Note here ,$\GL_1(\Z/n\Z)\subseteq \Aut(\Z/n\Z)$.











\section{Mar. 2}
We constructs $2$ non-abelian group of order $p^3$, where $p$ is an odd prime. One is of exponent $p$, the other is of exponent $p^2$

$\Aut(D_\infty)\cong D_\infty$, $\Out(D_\infty)=\Z/2/Z$, and $\Inn(D_\infty)\cong D_\infty$.

\begin{note}
If $H$ and $K$ are characteristic in $H\times K$, then $\Aut(H\times K)\cong \Aut (H)\times \Aut(K)$ as $\eta(h,k)=\phi(h)\psi(k)$. 	
\end{note}
\begin{example}[Non-example]
	$G=(\Z/p\Z)^k$, $\Aut(G)=\GL_k(\Z/p\Z)\supset \Aut(\Z/p\Z)\times...\times \Aut(\Z/p\Z) \cong (\Z/p\Z)^\times\times ...\times (\Z/p\Z)^\times$.
\end{example}
$\Aut(\Z/n\Z)=(\Z/n\Z)^\times=\{a+n\Z:\gcd(a,n)=1\}$ where $\phi_a(k)=ak$.
\begin{example}
	If $n=p_1^{k_1}...p_s^{k_s}$, then $\Z/n\Z\cong \Z/p_1^{k_1}\Z\times ...\times \Z/p_s^{k_s}\Z$ and each factor is characteristic, so $\Aut(\Z/n\Z)\cong \Aut(\Z/p_1^{k_1}\Z)\times...\times \Aut(\Z/p_s^{k_s}\Z)$.
\end{example}
\begin{note}
	What is $\Aut(\Z/p^k\Z)$? $|(\Z/p^k\Z)^\infty|=p^k-p^{k-1}$.
\end{note}
\begin{definition}
	The Euler's function $\phi(n)=|(\Z/n\Z)^\infty|$. We have $\phi(p_1^{k_1}...p_s^{k_s})=\phi(p_1^{k_1})...\phi(p_s^{k_s})$.
	
	If $\gcd(m,n)=1$, then $\phi(mn)=\phi(m)\phi(n)$.
\end{definition}
\begin{lem}
	1. If $k\geq 2$, then $\bar{5}\in(\Z/2^k\Z)^\times$ has order $2^{k-2}$.
	
	2. If $k\geq 1$, then $p+1\in (\Z/p^k\Z)^\times $ has order $p^{k-1}$.
\end{lem}
\begin{proof}
	2. if $K=1$, then $p+1=1$ has order $p^{k-1}$ in $(\Z/p^k\Z)^\times$
	
	Assume $p+1$ has order $p^{k-1}$ in $(\Z/p^k\Z)^\times$. Then $(p+1)^{pk-1}=1+Ap^k$ and assume $p\not| A$.
	
	Look at $(p-1)^{p^k}=[(p+1)^{p^{k-1}}]^p=(1+Ap^k)^p=1+\binom{p}{1}Ap^k+\binom{p}{2}A^2p^{2k}+...=1+p^{k+1}B$ for some $p\not|B$.
	
	From this, we have $(1+p)^{p^{k-1}}\equiv 1(\mod p^k)$ and $(1+p)^{p^{k-2}}=1+Ap^{k-1}\not\equiv 1(\mod p^k)$ since $p\not| A$.
\end{proof}
\begin{cor}
$(\Z/2^k\Z)^\times=\begin{cases}
	1&k=1\\\Z/2\Z&k=2\\\Z/2^{k-2}\Z\times\Z/2\Z=\langle \bar 5\rangle\times \langle\bar{-1}&k\geq 3
\end{cases}$	
\end{cor}
What about $(\Z/p\Z)^\times$?
\begin{thm}
	If $F$ is a field and $A\subseteq F^\times$ is a finite subgroup then $A$ is cclic.
\end{thm}
\begin{proof}
	Let $N$ be the exponent of $A$. So, $a^N=1$ for all $a\in A$.
	
	Recall that a polynomial of degree $k$ has at most $k$ roots in a field $x^N-1$ is of degree $N$ so $|A|\leq N$.
	
	$A$ abelian of exponent $N$, so $A$ has an element $a$ of order $N$ so $|A|\geq |\langle a \rangle|=N$. So, $a=\langle a\rangle$.
\end{proof}
\begin{cor}
$(\Z/p\Z)^\times$ is cyclic of order $p-1$; i.e., there is a $a\in \Z\st a,a^2,...,a^{p-1}$ are all distinct $\mod p$. Any such $a$ is called a primitive root module $p$. 	
\end{cor}
\begin{thm}
	$(\Z/p^n\Z)^\infty$ is cyclic for odd primes $p$, $n\geq 1$.
\end{thm}
\begin{proof}
	$(\Z/p\Z)^\times\twoheadrightarrow(\Z/p\Z)^\times$ and any $b$ which maps to a generator has order divisible by $p-1$ so some power of $b$ has order $(p-1)$. Here, $(\Z/p^n\Z)^\times$ has an element $u$ of power $p-1$ and an element $w=1+p$ of order $p^{n-1}$.
	
	So, $uw$ has order $p^{n-1}(p-1)=\phi(p^n)$. So, $(\Z/p^n\Z)^\times=\langle uw\rangle$.
\end{proof}
\begin{thm}[Euler]
	If $\gcd(a,n)=1$, then $a^{\phi(n)}\equiv 1(\mod n)$. Here, $|(\Z/n\Z)^\times|=\phi(n)$.
\end{thm}
\begin{example}
	$(\Z/20\Z)^\times\cong(\Z/4\Z\times\Z/5\Z)^\times\cong\Z/2\Z\times\Z/4\Z$. Notice $\phi(20)=8$.
\end{example}
\begin{definition}
	A representation of a group $G$ is a homomorphism $\phi:G\to \Aut(M)$ where $\Aut(M)$ are ``symmetries'' (or ``automorphism'') of some sort of object. A presentation is faithful if $\phi$ is injective.
\end{definition}
\begin{example}
	$M$ a vector space over a field.
	
	$\phi:G\to GL(M)$ where $GL(M)$ is the group of all invertible linear maps $M\to M$ are linear representations.
\end{example}
\begin{example}
	$M$ is a metric space, then $\Aut(M)$ are isometries of $M$.
\end{example}
\begin{example}
	Permutation representations is $G\to\Sym(X)=S(X)$ where $S(X)$ is the group of all permutations of $X$.
\end{example}
\section{Mar. 4}
\begin{definition}
	A \emph{permutation representation} of a group $G$ on a set $X$ is a homomorphism $\pi:G\to \Sym(X)$. $\Sym(X)=S(X)$ is the permutation of $X$. 
	
	We call a representation faithful if it is injective.
\end{definition}
\begin{definition}
	Given a representation $\pi:G\to S(X)$, we define a function $\star:G\times X\to X$ ($(g,x)\to g\star x$) by $g\star x=\pi(g)(x)$.
	
	It has 2 properties:
	\begin{enumerate}
		\item $g\star(h\star x)=(gh)\star x$	
		\item $e\star x=x$
	\end{enumerate}
\end{definition}
\begin{proof}[Proof of property 1]
	We have $$g\star(h\star x)=g\star(\pi(h)(x))=\pi(g)(\pi(h))x=\pi(gh)(x)=(gh)\star x$$
\end{proof}
\begin{definition}
	Any function $\star:G\times X\to X$ with properties 1 and 2 is called a \emph{left group action} of $G$ on $X$.
\end{definition}

Conversely, let $\star:G\times X\to X$ be an action of $G$ on $X$.

For $g\in G$, define $L_g:X\to X$ by $x\mapsto g\star x$.

Then, by 1, we have $L_g\circ L_h=L_{gh}$ and by 2, we have $L_e=id$; in particular, $L_g\circ L_{g^{-1}}=L_{gg^{-1}}=L_e=id=L_{g^{-1}}=L_g$.

So, each $L_g$ is a bijection. Therefore, $\pi:G\to S(X)$ by $g\to L_g$ is a homomorphism and we get a permutation representation.

We thus conclude that permutation representation and actions are essentially the same thing.
\begin{note}
	Let $G$ act on $X$. We write $gx$ instead of $g\star x$ whenever there are no confusions.
\end{note}
\begin{definition}
	For $s\in X$, the \emph{orbit} of $s$ is the set $O(s)=\{gs:g\in G\}$.
\end{definition}
\begin{prop}
	If $s,t\in X$ then either $O(s)=O(t)$ or $O(s)\cap O(t)=\emptyset$. 
\end{prop}
\begin{proof}
	If $v\in O(s)\cap O(t)$, then $v=as=bt$ for some $a,b\in G$. $O(g)\ni gs=(ga^{-1})(as)=(ga^{-1})(bt)=(ga^{-1}b)t\in O(t)$.
	
	Similarly, we have $O(t)\subseteq O(s)$. So, $O(s)=O(t)$.
\end{proof}
\begin{cor}
The orbits of an action on $X$ partition the set $X$.	
\end{cor}
\begin{definition}
	The \emph{stabilizer} of  $s\in X$ is the set $\St(s)=\{g\in G:gs=s\}$.
\end{definition}
\begin{prop}
	\begin{enumerate}
		\item $\St(s)$ is a subgroup of $G$.
		\item $\St(gs)=g\St(s)g^{-1}$.
	\end{enumerate}
\end{prop}
\begin{proof}
	If $h\in\St(s)$, then $(ghg^{-1})(gs)=gh(s)=gs$. The converse is easy to see.
\end{proof}
\begin{definition}
	Let $G$ act on $X$. For $Y\subseteq X$, define:
	\begin{enumerate}
		\item Stabilizer of $Y$, $\St(Y)=\{g\in G:gY=Y\}=\{g\in G:gY\in Y,g^{-1}y\in Y\text{ for all }y\in Y\}$
		As an exercise show that $\St(Y)$ is a subgroup of $G$.
		\item the point-wise stabilizer of $Y$, $G_Y=\{g\in G:gy=y\text{ for all }y\in Y\}=\bigcap\limits_{y\in Y}\St(Y)$.
	\end{enumerate}
\end{definition}
Note that $\St(s)=\St(\{s\})=G_{\{s\}}$ and we sometimes denote it as $G_s$.

Also, note that $\St(Y)$ acts on $Y$.
\begin{definition}
	$Y$ is $G$-stable if $\St(Y)=G$.
\end{definition}
\begin{note}
\begin{enumerate}
	\item Every orbit is $G$-stable
	\item $Y$ is $G$-stable iff it is a union of some collection of orbits.
\end{enumerate}	
\end{note}
\begin{definition}
	The action is \emph{transitive} if it has only one orbit.
\end{definition}
\begin{definition}
	Two actions of $G$ on $X$ and $Y$ are \emph{equivalent} if there is a bijection $f:X\to Y\st f(g(x))=g(f(x))$ for all $x\in X$. 
\end{definition}
Question: What does it mean in terms of representations?

Show as an exercise that for a subgroup $H\leq G$, we define $G/H$ to be the set of all left cosets of $H$ in $G$. Then we have an action of $G$ on $G/H$ by $g(aH)=(ga)H$ and this action is transitive with $\St(eH)=H$.

\begin{thm}
	Given an action of $G$ on $X$ and $s\in X$, the action of $G$ on $O(s)$ is equivalent to the action of $G$ on the left cosets of $\St(s)$.
\end{thm}
\begin{proof}
	Consider a map $G/\St(s)\to O(s)$ by $g\St(s)\to gs$.
	
	This map is well defined as if $g\St(s)=g_1\St(s)$, then $g_1s=gh$ for some $h\in \St(s)$ and so $g_1s=(gh)s=g(hs)=gs$.
	
	It is also clear that this map is surjective.
	
	Now, suppose that $gs=g_1s$, then $(g_1^{-1}g)s=s$, so $g_1^{-1}g\in \St(s)$. Therefore, $g_1\St(s)=g\St(s)$. So, it is bijective.
	
	Notice that $\phi(g(a\St(s)))=\phi(ga\St(s))=(ga)s=g(as)=g(\phi(a\St(s)))$.
\end{proof}
\begin{cor}
	$|O(s)|=[G:\St(s)]$, and if $G$ is finite, then $|O(s)=\frac{|G|}{|\St(s)|}|$.
\end{cor}
\section{Mar. 7}
\begin{thm}
	Let a group $G$ act on a set $X$. For any $s\in X$, the action of $G$ on the orbit $o(s)$ is equivalent to the action of $G$ on the left cosets of $\St(s)$; i.e., $G/\St(s)$.
	
	In particular, $|O(s)|=[G:\St(s)]$. If $G$ is finite then $|O(s)|=\frac{|G|}{|\St(s)|}$.
\end{thm}
\begin{cor}
	Any translation action is equivalent to the action of $G$ on $G/H$, with left multiplcaition for some $H\subseteq G$.
\end{cor}
\begin{note}
	In the action of $G$ on $G/H$, we have
	\begin{enumerate}
		\item $\St(eH)=H$
		\item the kernel of the action, $\bigcap\limits_{g\in G}gHg^{-1}$ is the largest normal subgroup of $G$ contained in $H$.
	\end{enumerate}
\end{note}
Show as an exercise that the action of $G$ on $G/H$ and $G/K$ are equivalent iff $H$ and $K$ are conjugate in $G$.
\begin{definition}
	A point $s\in X$ is called a \emph{fixed point} if $O(s)=\{s\}$; i.e., $\St(s)=G_s=G$.
\end{definition}
\begin{definition}
	For any subset $Y\subseteq G$, the fixed pints of $Y$, $\Fix(Y)=\{s\in X:gs=s\text{ for all }g\in Y\}$
\end{definition}
As an exercise, show that $\Fix(Y)=\Fix(\langle Y\rangle)$.

\begin{note}
	For $G$ acting on $X$, we have
	\begin{enumerate}
		\item if $H\leq G$ then $H$ acts on $X$.
		\item If $Y\subseteq X$ is $G$-stable ($\St(Y)=G$), then $G$ acts on $Y$.
		\item this action induces an action on the power set of $X$, $P(X)$ (the set of all subsets of $X$) by $g\cdot Y=\{gy:y\in Y\}$ (Note that $g\emptyset=\emptyset$).
		\item For each $k\leq |X|$, the set of all subsets of size $k$, $P_k(X)$ is $G$-stable, so $G$ acts on $P_k(X)$
	\end{enumerate}
\end{note}
\begin{example}
	$G=S_n$ acts on $X=\{1,2,...,n\}$, so it acts on $P_k(X)$. This action is transitive so it extends to a partition of $X$.
	
	We note that $\St(\{1,2...,k\})\cong S_k\times S_{n-k}$. So, $|P_k(X)|=\frac{|S_n|}{|S_k\times S_{n-k}|}=\frac{n!}{k!(n-k)!}=\binom{n}{k}$.
\end{example}
Let $n=p^km$, $p$ a prime, $k>9$. Let $\pi\in S_n$ be 
$$\begin{pmatrix}
	1&\dots&p^k&p^k+1&\dots&2p^k & \dots &(m-1)p^k+1&\dots&mp^k\\
	2&\dots &1&p^k+2&\dots &p^k+1&\dots &(m-1)p^k+2&\dots&(m-1)p^k+1
	
\end{pmatrix}$$
We note that $|\pi|=p^k$ and $\pi^m(1)=m+1$ for $m<p^k$. So, $\langle \pi\rangle\subseteq S_n$ has order $p^k$ and acts on $P_{p^k}(X)$.  What are the fixed points of this action? THe fixed points are exactly the fixed points of $\pi$ and are $\{1,2,...,p^k\}, \{p^{k+1},...,2p^k\},\{(m-1)p^k+1,...,mp^k\}$.

Every orbit of $\langle \pi\rangle $ other than the fixed points on $P_{p^{k}}(X)$ will have size a positive power of $p$, hence divisible by $p$. so $|P_{p^k}(X)|=m+Ap$?

Thus, $\binom{p^km}{p^k}\equiv m(\mod p)$.

Give a ``direct'' proof of this thm as an exercise.
\begin{cor}
	If $p\not|m$, then $p\not| \binom{p^km}{p^k}$.
\end{cor}
\begin{note}
	Let $|G|=p^km$, $p\not|m$, $p$ a prime, $k>0$. Then $G$ acts on itself by left multiplication $X=G=G/\{e\}$, so it acts on $P_{p^k}(G)$. But $p\not| \binom{p^km}{p^k}=|P_{p^k}(G)|$.
	
	So at least one orbit $O(A)$ has size not divisible by $p$. If $p\not|O(A)$, then $|\St(A)|=\frac{G}{O(A)}\geq p^k$, but $|\St(A)|\leq |A|=p^k$ as if $a\in A$, then $\St(A)\cdot a\subseteq A$.
	
	Thus, $|\St(A)|=p^k$.
\end{note}
\begin{thm}[Sylow]
	If $|G|=p^km$ and $p\not|m$, then $G$ has a subgroup of order $p^k$.
\end{thm}
Any such subgroup is called a \emph{Sylow $p$-subgroup} of $G$.

We have three basic rules for a finite group $G$ acting on a finite set $X$.
\begin{thm}
	We have three basic rules for a finite group $G$ acting on a finite set $X$.
	\begin{enumerate}
		\item If $G$ acts transitively on $X$, then $|X|=\frac{|G|}{|\St(s)|}$ for any $s\in X$.
		\item If $p$ is a prime and $p\not||X|$, then $p\not||O(s)|$ for some $s\in X$.
		\item If $|G|=p^r$, $p$ a prime and $r>0$ amd $|\Fix(G)|=f$, then $|X|\equiv f(\mod p)$.
	\end{enumerate}
	In particular, if $p\not||X|$, then $f>0$, so there is a fixed point. and if $p||X|$ and $f>0$, then $f>p$ so we have at least $p$ fixed points.
\end{thm}
\begin{thm}[Cauchy]
	If $G$ is a finite group $p||G|$ with $p$ a prime. Then $G$ has an element of order $P$.
\end{thm}
\begin{proof}
	Let $|G|=p^km$, $k>0$ and $p\not|m$. Then $G$ has a subgroup $P$ of size $p^k$. Take $1\not=a\in P$. Then $O(a)$ is a power of $p$. So some power of $a$ has order $p$.
\end{proof}
\begin{definition}
	A group $P$ is a \emph{$p$-group} if every element of $P$ is of finite order = power of $p$
\end{definition}
\begin{cor}
	A finite group is a $p$-group iff $|G|$ is a power of $p$.
\end{cor}
\section{Mar. 9}
\begin{thm}
	Let $G$ be a finite group acting on a finite set $X$.
	\begin{enumerate}
		\item If $G$ acts transitively on $X$, then $|X|=\frac{|G|}{|\St(s)|}$ for any $s\in X$.
		\item If $p$ is a prime and $p\not||X|$, then $p\not||O(s)|$ for some $s\in X$.
		\item If $|G|=p^r$, $p$ a prime and $r>0$ amd $|\Fix(G)|=t$, then $|X|\equiv t(\mod p)$.
	
	In particular, if $p\not||X|$, then $t>0$, so there is a fixed point. and if $p||X|$ and $t>0$, then $t\geq p$ so we have at least $p$ fixed points.
		\item $|X|=\sum_{\text{orbits O}}|O|=\sum_{\text{orbits}}\frac{|G|}{|\St(s)|}$.
		\end{enumerate}
\end{thm}
\begin{thm}
	If $|G|=p^km$, $p$ a prime, $k>0$ and $p\not|m$, then $G$ has a subgroup of order $p^k$.
\end{thm}
\begin{thm}[Cauchy]
	If $G$ is a finite group, $p||G|$ with $p$ a prime. Then $G$ has an element of order $P$.
\end{thm}
\begin{cor}
	A finite group is a $p$-group iff the size of $P$, $|P|$ is a power of $p$.
\end{cor}
\begin{note}
	We consider the following ``key'' action.
	
	Any group $G$ acts on itself by conjugation: $g\star s=gsg^{-1}$ ($G\to \Aut(G)\subseteq S(G)$).
	
	$G$ acts on $P_{k}(G)$ for all $k$. The set of fixed points are normal subgroups. 
	
	Orbits of this action on $G$ are called \emph{conjugacy classes} and fixed points are exactly those elements in the center of $G$.
\end{note}
\begin{definition}
	For $X\in G$, the \emph{normalizer} of $X$ in $G$, $N_G(X)=\St(X)$ under the conjugation action.
	
	The \emph{centralizer} of $X$ in $G$, $C_G(X)=G_X$.
\end{definition}

If $H\leq G$, then $H$ is normal in $N_G{H}$.

In particular, $G$ acts on the set $\Syl_p(G)$ of all sylow $p$-subgroups of $G$.

\begin{note}
	Let $|G|=p^km$ where $p$ a prime, $k>0$, $p\not|m$. Then $G$ acts on $\Syl_p(G)$ by conjugation.
	
	Take $P\in \Syl_p(G)$ and $Q\leq G$ where $Q$ is some power of $p$.
	
	$Q$ acts on the orbit $O(P)$.
	
	Take a $G$-orbit of $P$, $O(P)$. We have
	\begin{enumerate}
		\item $\St(P)=N_G(P)\supseteq P$, so $p^k||N_G(P)|$ and since $|O(p)|=\frac{|G|}{|N_G(P)|}$, so we have $p\not\vert |O(P)|$
		\item consider the action of $Q$ on $O(P)$. $|Q|$ is some power of $p$ and $p\not||O(P)|$, so there exists a fixed point.
	\end{enumerate}
	
	Then, we can take $P_1\in O(P)\st Q$ fixes $P_1$; i.e., $Q\leq N_G(P_1)$ so $Q\subseteq P_1$.
\end{note}
\begin{cor}
	If $Q\in\Syl_p(G)$, then $Q=P_1$; i.e., $Q\in O(P)$.
\end{cor}

So, $G$ acts transitively on $\Syl_p(G)$.

Also, $Q$ has only 1 fixed point: $Q$ itself. So, $|\Syl_p(G)|\equiv 1 (\mod p)$.
\begin{thm}[Sylow]
	Let $|G|=p^km$ where $p$ a prime, $k>0$, $p\not|m$.
	\begin{enumerate}
		\item $G$ has at least one subgroups of order $p^k$ (Sylow $p$-subgroup).
		\item All Sylow $p$-subgroups are conjugate.
		\item Let $t_p=|\Syl_p(G)|$. Then $t_p\equiv 1(\mod p)$ and $t_p|m$.
		\item Any $p$-subgroup of $G$ is contained in a Sylow $p$-subgroup.
	\end{enumerate}
\end{thm}
Note that $t_p=1$ iff $G$ has a normal Sylow $p$-subgroup.
$$P\hookrightarrow G\twoheadrightarrow G/P$$
and $G\cong P\rtimes G/P$
\begin{example}[Groups of order $pq$, $p<q$ primes]
	Let $|G|=pq$. Let's look at $\Syl_q(G)$. Since $t_q|p$ and $t_q\equiv 1(\mod q)$, we have $t_q=1$ as $p<q$.
	
	So, $G$ has a normal Sylow $p$-subgroup $Q$. $Q$ is cyclic of order $q$.
	
	By Cauchy, $G$ has an element $a$ of order $p$ and $H=\langle a\rangle\in\Syl_p(G)$ is a cyclic subgroup of order $p$ in $G$. Look at $t_p$, if $t_p=1$, then $H\unlhd G$, so $G\cong Q\times H$.
	
	If $t_p=q$, then $t_p=q\equiv 1(\mod p)$.
	
	So if $q\not\equiv 1(\mod p)$, then the only finite group of order $pq$ is $\Z/p\Z\times\Z/q\Z\cong\Z/pq\Z$.
	
	If $q\equiv 1(\mod p)$, then $\Aut(Q)=\Aut(\Z/p\Z)\cong (\Z/p\Z)^\times$ which is the cyclic group of order $q-1$. So it has a unique subgroup of order $p$ as ($p|q-1$).
	
	Take $\phi:H=\Z/p\Z\hookrightarrow\Aut(\Z/q\Z)$. Then $Q\rtimes_\phi H$ is a non-abelian group of order $pq$.
\end{example}

As an exercise, show that all possible $\phi$ gives the same group $G=\Z/p\Z\rtimes_\phi \Z/q\Z$.
\begin{cor}
	Every group of order $p^2$ is abelian.
\end{cor}

Our goal is to study $p$-groups.

$P$ a $p$-group. It acts on itself by conjugations. $e$ is a fixed point, so $|Fix(P)|\geq p$. Here, $\Fix(P)=Z(P)$.
\end{document}
